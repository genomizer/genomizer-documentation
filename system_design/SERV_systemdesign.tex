The \appName\ service needs to be able to convert, process and visualize data. This chapter explains how this is done in the system.

%\begin{figure}[h]
%\addImage{UMLFinal.png}
%\caption{Class-diagram  for Process}
%\label{con_UML}
%\end{figure}
	
As can be seen in \refer{con_UML} the RawToProfileConverter, Smooth and Step extends the Executor class. The different processing commands can only start the corresponding processing method on these Executors. 


\subsubsection{Executor}
The executor class, as seen in figure 5.2.1, is a abstract superclass that is an entity that is able to execute various commands. The executor class is able to run programs as well as scripts and shell commands. The commands are specified in the call to the methods in this class. \newline

\begin{tabularx}{\textwidth}{|l|X|}
\hline
\term{executeCommand} &

\term{executeCommand} is a protected method used in processing to make command line calls to external dependencies used
in the various processing steps. Firstly a \term{processBuilder} is used to ensure a safe way to execute commands, after 
that the working directory is set and the error output stream is merged with the standard output. After a command has been 
started the output stream is then recorded with the help of a Scanner object and a \term{stringBuilder} object. When the 
command has been executed the termination status is checked and the recorded string is sent back to the caller. The command 
to be executed is represented as an array of strings.
\\ \hline

\end{tabularx}

\subsubsection{RawToProfileConverter}
The purpose of the \term{RawToProfileConverter} class is that it will be used by
\term{RawToProfProcessCommand} and do all the steps in the process pipeline produce a \term{profile
file} in \term{.wig} format. These steps are done by calling external dependencies such as programs and scripts which are executed with methods that is extended from
\term{Executor} class. 
\subsubsection{Smoothing}
The term{Smoothing} class is used on a \term{profile file} to smooth down the tips, making the data result less jagged.
\subsubsection{Step}
The term{Step} class is used on a \term{profile file} to lower the file's resolution.
\subsubsection{Description of different scripts and processing steps}

\begin{enumerate}
\item \term{BowTie}: Uses the external dependency Java tool Bowtie. 
Support for Bowtie2 is implemented but not fully tested. 
Bowtie creates unsorted \term{.sam} files from \term{.fastq} raw files.
The files are created in a temporary folder with the name \filePath{result\_X}, where X is the ID of the current thread. 
All other folders created is placed inside the folder from where the files used where placed.

\item \term{sortSam}: Uses external dependency Picard and sorts the \term{.sam} file and creates a new \term{.sam} files, sorted by coordinates.
The files are saved in the same temporary folder as in the Bowtie step.

\item \term{RemoveDuplicates}: Uses the external dependency Python tool Pyicos.
Takes a sorted \term{.sam} file and produces a new \term{.sam} file with all duplicate reads removed.
It is optional to save this \term{.sam} file to the database but it is saved in the temporary directory in the mean time.

\item \term{Convert}: Uses external dependency Python tool Pyicos.
This is the final step of raw to profile conversion and uses Pyicos to convert a given \term{.sam} file to \term{.wig} file.
All intermediate files are removed except optionally the \term{.sam} file which can be returned together with the final \term{.wig} file. All saved files are moved to the given profile directory path.

\item \term{Smooth}: smooths the file and creates a large \term{.sgr} file,
converted the customers \term{Perl script} by following the algorithm they  sent
us. This makes it more efficient. Puts the files in a folder called
\filePath{smoothing}.

\item \term{Step}: Takes the smoothed \term{.sgr} file and takes samples from it
with a specified interval and creates a smaller \term{.sgr} file. If stepping is done the files will be placed in the same folder as the previous step.

\item \term{Ratio Calculation}: Creates four \term{.sgr} files with the
\term{Perl script}
provided by the customer. Puts the files in a folder called \filePath{ratios}.

\item \term{Smooth}: After the ratio calculation, smoothing needs to be done
again with different parameters. Puts the files in a folder called
\filePath{smoothing}
\end{enumerate}


\LTXtable{\textwidth}{system_design/SERV_systemServer_ProcessingTable.tex}

\paragraph{BowTie}
BowTie takes a raw \term{.fastq} file together with a genome release and converts the \term{.fastq} file to a \term{.sam} file, which is the first step to make the desired \term{.wig} file.
After a \term{.sam} file is converted the external dependency \term{Picard} is run with its internal command \term{samSort}, which produces a sorted \term{.sam} file sorted by chromosome and position as needed to use the scripts.
\paragraph{After-processing scripts}
The different functions of the Perl scripts is explained below. They are explained in the same order that they are executed. All scripts take a directory of files to be processed as input parameter.
The given Perl scripts are modified and wrapped by expect scripts in order for better usability and callability from the Java implementation.
\subsubsection{Ratio calculation}
Does ratio calculation on the processed files, for each position in the IP sample with at least one mapped read, a ratio of IP - input (on a log2 scale) is calculated. If the read count in the input is below the read count mean (in the input sample) is calculated it is set to the mean ( or double mean (2 x mean) as user specified). If the input mean is below four the minimum input value is set to four (to avoid division by near zero values. Calculated as (read length x approximate total number of reads in input samples(9 millin))/ genome size (for Drosophila melanogaster 120381546)). A random number between -0.5 and 0,5 is added to the read counts before log2 conversion to make them discrete for statistical analysis. All ratio values are then adjusted by reducing each value by median of the ratios. This linear adjustment is carried out in order to compensate for differences in IP and input sequencing depth. Also, to visualize ratios distribution, ratios are plotted by binning ratios with user specified numbers of bins and minimum and maximum ratio values (200bins,minimum ratio value: -10, maximum ratio value:10). Ratio values are printed in sgr format.

\subsubsection{Smoothing and stepping}
Both Smoothing and Step are implemented as separate classes calling external Perl scripts.
The classes provide some validation and a clean interface towards the external dependencies.
The programs can handle file corruption to some extent. 
If the file contains empty or wrongly formatted rows the program will not crash, it will simply ignore the corrupt rows.

\paragraph{Smoothing}
Smoothing means that a trimmed mean value or median value for a position and its surrounding positions is calculated. The number of positions to smooth on is called the Window Size. For example: with a window size of 10, the smoothed value on position X is calculated on the interval (X-4, X+5). A number of position which below shouldn't be smoothed at all should also be provided. There's also one parameter called stepSize, if the stepSize is one the program will not do any stepping but if it's larger than 1 stepping will be done. Stepping is handled in this program by simply checking every time we are going to write to the new file if the current row's position is divisible with the stepSize, if it is we write to the file, otherwise the row is discarded.

\paragraph{Step}
Step also takes a window size, the number of genome reads to skip. 
This afterprocessing reduces the granularity of the file and thus the file size, whilst information is lost of course.


\paragraph{Tuple}
The tuple class is a data carrier that represents one row of data in an sgr file. It consists of the fields chromosome, position, signal and newSignal. Where signal is the signal-value read from the infile and newSignal is the updated value after smoothing have been done.
The methods in this class are all standard getters/setters except for the method toString which formats a row for the outfile and rounds of decimal numbers. The constructor is also of interest since it parse a row on tabs. Thus the fields in an infile needs to be seperated by tabs and not spaces. The constructor will throw an exception if the line it tries to parse is either null or if it does not consist of three columns separated by tabs where the first is a string and the second and third is a double.


\subsubsection{ProcessHandler}
The ProcessHandler is a controller that handles process-calls. Depending on the name of the process it handles it differently. It acts as an interface between the process-module and the rest of the program. 


\subsubsection{Logic \& interface}
The main logic in the ProcessHandler is a switch-case that switches on the name of the process being called. For example if the name of the process is “RawToProfile” is sets up a RawToProfile-converter and calls it. 

\begin{tabular}{|l| p{7cm}|}
\hline
processName & A string that tells the handler which kind of process should be
executed. \\ \hline
procedureParams & A list of string with the parameters to the different external
programs/scripts that will be called during the execution. The first element
will be a string with parameters/flags for the first external program that will
be called, and so on. \\ \hline
inFile & A string with a path to the directory containing the files that should
be operated on. \\ \hline
outFile & A string with a path to the directory where the result .sgr files
should be put. \\ \hline
\end{tabular}



:
