
\subsubsection{Overview of the desktop client}
The UML diagram in \refer{fig:des_uml-overview} describes the whole desktop client.
\begin{figure}[htb!]
	\addImage{UMLdesktop.jpg}
	\caption{UML diagram over the desktop client}
	\label{fig:des_uml-overview}
\end{figure}


\subsubsection{View}
The view of the Genomizer Desktop client is constructed around tabs. There are 6 different tabs. These are Search, Process, Upload, Workspace, Analyse and Administration. As of now the Analyze tab is not implemented.

Each tab in the view is represented by its own java class. The QuerySearchTab class which represents the search tab can display both a search view and a results view. It uses the QueryBuilderRow class to construct the rows in the query builder which is used to construct search queries. The QueryBuilderRow class represents a row in the query builder and each row is dynamic and can change accordingly to user interaction.

The search results are also implemented in the QuerySearchTab and the results are displayed with the TreeTable class which is further described in the utilities section below.

The UploadTab Class represents the upload view of the GUI. It has functionality to both upload a file to an existing experiment (which is separately handled in the UploadExistingExpPanel) and to create a new experiment to upload files to.

The ProcessTab class represents the process view in the GUI. It contains a list where files to be processed can be stored and buttons and eight parameters for initiating the processing. There is also a function to retrieve current processes from the server and display them in a tree.

The WorkspaceTab class consists of four buttons and a TreeTable that holds all the experiments and their data. The buttons are:  Remove selected, Download selected, Analyze selected and Process selected. From the workspace, the download function/window is accessible. The DownloadWindow holds the FileData to be downloaded and its GUI consists of a JTable showing the file names and has JComboBoxes for choosing file format and a download button which opens a JFileChooser to download files.

The AnalyzeTab Class is not yet implemented.

\subsubsection{Model}
The model part of the system contains method for doing most of the logic in the system. For example there are methods for sending login requests and for downloading files. There are separate classes for downloading and uploading files as well as a class for regular communication with the server called Connection. New connections are created with the ConnectionFactory class.

\subsubsection{Requests}
The Request package contains the Request class , the RequestFactory and all the classes that extends the Request class. Request is the super class and can make a JSON package that all the other Request classes can use. All requests must have a name, type and an URL, but can consist of more information. For example LoginRequest also has username and password. RequestFactory is a class that can create all objects from all types of requests. It is a way to easily create all requests from the same place.


\subsubsection{Response}
This package consists of all types of responses that the server can send to the client-program. There is a class named Response that all the other response classes extends from. For example there is a response class for the login request called LoginResponse. All types of responses have different properties. There is also a class ResponseParser that can parse the responses so that the important information can be taken out of a JSON-package. This information can then be used to tell the client program what should happen next in the user interface.


\subsubsection{Controller}
The controller part of the system consists of ActionListeners for the different buttons and functionalities in the view. For example there are Listeners for searching, downloading and processing. The Controller class has access to both the view and the model and acts as a middle hand between those two parts of the system. Usually a Listener in the controller reacts upon user input and then modifies the model and gives information about the change to the view.


\subsubsection{Utilites}

There are several classes which represents different data in the system. There are classes for experiment data, file data and annotation data. For example when a search response is received from the server it is parsed into experiment data and the experiment data contains file data and annotation data. There is also a class representing Process feedback data.

The TreeTable class represents the table which displays experiment data, annotation data and file data in the Search and Workspace tabs. It is specially constructed to handle the data classes and it allows vertical sorting.

\subsubsection{System Administration}
%Till Sysadmin!

\paragraph{Communication with the Server}
\label{Communication with the Server}

All communication between the server and the system administration tab follows a line of steps. See \refer{fig:adm_com_view} below.

\begin{enumerate}

  \item An event is triggered by the user clicking something.
  \item The listener for the active tab receives the event and sorts out which type it is, and calls the appropriate method in the \textit{SysadminController}.
  \item The \textit{SysadminController} has the connection to the \textit{Model}, and calls the associated method there.
  \item The \textit{Model} creates the corresponding request for the server, and then creates a new connection.
  \item The \textit{Connection} receives the request from the \textit{Model} and sends the request to the server.


\end{enumerate}

If the event triggers a request for data, the\textit{Model} will use a parser to parse the data before sending it back to the GUI to present it to the user.


\begin{figure}[hbt!]
\addImage{sysad_comm_view.png}
\caption{Communication Overview}
\label{fig:adm_com_view}
\end{figure}


\paragraph{A communication example}
\label{Communication example}

As an example, assume that the user clicks the 'Genome Files' tab.
This triggers the \textit{SysadminTabChangeListener} to receive an event. The desired behavior of the tab is to directly show the available genome releases, so now they  have to be fetched from the server. The \textit{SysadminTabChangeListener} therefore calls the \textit{SysadminController}.
This class then retrieves the \textit{GenomeReleaseTableModel} to be able to use it when sending the data to the user view.  After that it calls the \textit{getGenomeReleases()} method in the \textit{Model}. This method creates a \textit{GetGenomeReleaseRequest} to be sent to the server by using the \textit{RequestFactory} class. The \textit{Model} then creates a new \textit{Connection} by using the \textit{ConnectionFactory}. The request is then sent to the server. The \textit{Connection} receives the result and the \textit{Model} can read from it. In this case the response will be a JSON string containing all the the genome releases on the server. This string needs to be parsed into something more useful and thats when the \textit{ResponseParser} is used. It uses the Google Java library Gson, which is used to convert a JSON string into a Java object. In this case the \textit{ResponseParser} will convert the response JSON string into an array of \textit{GenomeReleaseData} objects. This array is then returned back by to the \textit{SysadminController} which updates the \textit{GenomeReleaseTableModel} with the \textit{GenomeReleaseData} objects. And the user can now see all available genome releases on the server.

\paragraph{Building the Tabs}
\label{Building the Tabs}

Each tab or toolset in the administration view is build in SysadminTab, as a panel and then added to the JTabbedPane. Every tab is in a own package, like annotation view and genomereleaseview. Each tab is build in smaller methods in each View and listeners to buttons and such are added in the SysController class.


\subsubsection{Flow of the system}

The sequence diagram in \refer{fig:des_download-sequence} describes the flow of the system when the user presses the download file button and the diagram in \refer{fig:des_login-sequence} describes how the desktop clients reacts to a login.

\begin{figure}[htb!]
	\addImage{download-sequence.jpeg}
	\caption{UML sequence diagram of downloading a file}
	\label{fig:des_download-sequence}
\end{figure}

\begin{figure}[htb!]
	\addImage{login-sequence.jpeg}
	\caption{UML sequence diagram of login}
	\label{fig:des_login-sequence}
\end{figure}
\FloatBarrier
