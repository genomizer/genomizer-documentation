\label{chap:com_latexoverview}
\section{System overview}
\label{sec:sys_overview}
%Things that people perceive as hard to change;[6] since designing the architecture takes place at the beginning of a software system's lifecycle, the architect should focus on decisions that “have to” be right the first time, since reversing such decisions may be impossible or prohibitively expensive.
The \appName\ is a server-client architecture. It consists of four different clients: a web client, a desktop client and a two kinds of mobile clients. The mobile clients are implemented in \textit{Android} and \textit{iOS}. The server side consists of three components, a web server that acts as a proxy, a \textit{Java} server that connects to the final component a \textit{postgresql} database.

The web server is a \textit{Apache} system in the current implementation.

\begin{figure}[h!]
\centering	
\begin{tikzpicture}[ 
	font=\sffamily,
	every matrix/.style={ampersand replacement=\&,column sep=0.5cm,row sep=0.5cm},
	db/.style={cylinder, shape border rotate=90, draw, fill=orange!40, minimum height=2cm, minimum width=1.5cm},
	server/.style={rectangle, draw, fill=orange!40, minimum height=1cm, minimum width=2cm, font=\ttfamily\footnotesize},
	webb/.style={rectangle, draw, fill=blue!40, minimum height=1cm, minimum width=2cm, font=\ttfamily\footnotesize}, 
	mobile/.style={rectangle, draw, fill=red!40, minimum height=1cm, minimum width=2cm, font=\ttfamily\footnotesize},
	desktop/.style={rectangle, draw, fill=yellow!40, minimum height=1cm, minimum width=2cm, font=\ttfamily\footnotesize},
	apache/.style={rectangle, draw, fill=green!40, minimum height=1cm, minimum width=2cm, font=\ttfamily\footnotesize},
	both/.style={<->,>=stealth',shorten >=1pt,semithick,font=\sffamily\footnotesize},
	to/.style={->,>=stealth',shorten >=1pt,semithick,font=\sffamily\footnotesize},
	every node/.style={align=center}]  
\matrix{
	\node[webb] (webb) {Web}; \&
	\node[desktop] (desktop) {Desktop}; \&
	\node[mobile] (mobile) {Mobile}; \\
	\& \node[apache] (webServer) {Web server\\ Proxy}; \& \\
	\& \node[server] (genomizer) {Genomizer}; \& \\
	\& \node[db] (db) {DB}; \& \\
	};
	
	\draw[both] (webb) -- (webServer);
	\draw[both] (desktop) -- (webServer);
	\draw[both] (mobile) -- (webServer);
	\draw[both] (webServer) -- (genomizer);
	\draw[both] (genomizer) -- (db);

\end{tikzpicture}
\caption{Overview of the system architecture}
\label{fig:archOverview}
\end{figure}

The connection to the server from any client goes through the web server which acts as a proxy as well. This to insure a \textit{SSL} connection between the web server and the clients and a regular connection between the server and web server. The proxy protects the web client from \textit{XSS} (Cross site scripting). 

All of the clients use the \term{RESTful} protocol where the message body consists of a \json\ object.
These messages are requests that gets passed through the web server to the \appName\ server. The \appName\ server communicates with the database to perform the request. Then it generates a response following the \term{RESTful} protocol. The \term{RESTful} protocol is used to promise that the server will be state-less. State-less meaning that a request from a client can not lock the server for other clients requests. 

The different requests that can be sent to the server are defined in the Appendix \fullref{chap:com_api} that goes into deeper detail about the API.

The architecture of the system can be seen in Figure \ref{fig:archOverview} where the clients are the colors blue, yellow, and red. Web server is green and the \appName\ server and its database are colored orange.

\subsection{Genomizer clients}

All the clients that are part of the \appName\ service are implemented using the architectural design pattern MVC (\texttt{Model View Controller}). This pattern is based around keeping the logic and data separate from its representation and user interface. The basic idea is shown in figure \refer{fig:MVCimage}.

\begin{figure}[h!]
\centering	
\begin{tikzpicture}[ 
	font=\sffamily,
	every matrix/.style={ampersand replacement=\&,column sep=1.5cm,row sep=1.0cm},
	db/.style={cylinder, shape border rotate=90, draw, fill=orange!40, minimum height=2cm, minimum width=1.5cm},
	server/.style={rectangle, draw, fill=orange!40, minimum height=1cm, minimum width=2cm, font=\ttfamily\footnotesize},
	human/.style={rectangle, draw, fill=white, minimum height = 1cm, minimum width = 1cm}, 
	both/.style={<->,>=stealth',shorten >=1pt,semithick,font=\sffamily\footnotesize},
	to/.style={->,>=stealth',shorten >=1pt,semithick,font=\sffamily\footnotesize},
	every node/.style={align=center}]  
\matrix{
	\node[server] (comm) {Communication}; \& 
	\node[server] (controller) {Controller}; \& 
	\\
	\& \& 	\node[human] (human) {\LARGE\Gentsroom\Ladiesroom}; \\	
	\node[server] (model) {Model}; \&
	\node[server] (view) {View}; \& \\
	};
	
	\draw (comm) -- (model);
	\draw[to] (controller) -- (model) node[midway, sloped, anchor=center, above] {\footnotesize\texttt{Update data}};
	\draw[to] (controller) -- (view) node[midway, xshift=0.2cm, yshift=-1.2cm, anchor=center, rotate=90, right] {\footnotesize\texttt{Listen to action}};
	\draw[to] (human) -- (view) node[midway, sloped, anchor=center, above] {\footnotesize\texttt{Interact}};
	\draw[to] (view) -| (human) node[midway, anchor=center, below] {\footnotesize\texttt{Show GUI}};
	\draw[to] (model) -- (view) node[midway, anchor=cneter, below] {\footnotesize\texttt{Represent}};
\end{tikzpicture}
\caption{Basic MVC pattern.}
\label{fig:MVCimage}
\end{figure}

 The view will contain lists of experiments and annotations, data regarding process status, and other information. It will also present the user with windows and buttons that present and allow data to be manipulated. When a user interacts with the graphical interface the action will be handled by the controller using listeners. This in turn modifies the model; representing the data and information, as well as the operations available on it. 
 
 For this project a communication part is also necessary; often as a part of the model. As mentioned in the overview section above, the communication is performed via a \term{RESTful} protocol using \json\ requests and responses. \json\ is used because it is easy to parse and is directly compatible with the standard frameworks used in the web client.
 
 The advantage of this approach lies in the separation of the interface- and logic-code; allowing one to be more easily replaced or modified. In our case the communication data and interface can be kept the same, while the client representation of it can be changed freely.
 

\subsection{Genomizer server}
The server is devided up in three main components, the front end, the data storage and the processing.

The front end consist of a handler for the \term{API} requests mentioned above. The handler interpret incoming request and passes them through to either processing or data storage depending request type.

Data storage makes up the back-end of the server. Its purpose is to communicate with the file system and the database to return the correct information that was requested by the front-end.

The processing component of the server has the purpose of interpreting the \term{raw} files and convert them to \term{profile} data. This process is a very heavy workload on the server need to be scheduled to promise efficiency. 

\refer{fig:com_systemoverview} shows a simple flow diagram which describes how the client and server
 communicates. The particular example shows the data flow when the client process a file.
 In the figure the front-end is the data transfer and communication node while data storage makes up the database node and processing the final process node.

\begin{figure}[htb]
\addImage{com_systemoverview.jpg}
\caption{A simple flow diagram for the system}
\label{fig:com_systemoverview}
\end{figure}

Every request the client does creates a non persistent connection to
the server. When the server receives a request it checks which kind of
request it is and routes it to the communication part of the server.

When the request is routed to communication a specific command is
created. The command is an object which consists of information from
the \term{RESTful}-header and \json/ body sent from the
client. The command is then parsed and sent to different parts of the
server, usually the database first, which returns information from a
\term{SQL query}. Depending on the requests this information can later
be used, for example, to process a file or be sent back to the clients
directly.

The clients are always going to receive a response code after each
request, but in some cases the respond also contains a \json/
body with information which can be shown to the user. This is the case
for requests like $getAnnotations$. The response can also contain
error messages, describing what went wrong when executing the command.

After a client receives the response the connection with the server is
closed and a new is opened with the next request.
