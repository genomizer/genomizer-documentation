% Here you can find a basic overview of the system in general.

\section{A system overview}

The \appName\ is a server-client system, which involves four different clients, a Java server and one postgresql database. The different kinds of clients are:

\begin{itemize}
\item iOS-client
\item Android-client
\item Web-client
\item Desktop-client
\end{itemize}


All of these clients use RESTful and Json to communicate with the server, sent over a non persistent HTTP-socket. How the different requests sent over this socket is specified in the API which can be found at : www........... 

Every request the client does creates a non persistent connection to the server. When the server receives a request it checks which kind of request it is and creates the corresponding internal command for it.

This command is an object which consists of information from the RESTful-header and Json body sent from the client. The command is sent to the database which returns information gotten by a SQL query. Depending on the requests this information can later be used to, for example processes a file or be sent back to the clients. The clients always going to receive a response code after each requests, but in some cases the respond also contains a Json body with information which can be shown to the user. This is the case for requests like getAnnotations.

After a client received the response the connections with the server disappears until the next request. 

There is a special kind of user called system admin. A user with these priveleges has the rights to add and delete annotations.

\begin{figure}[t]
\addImage{com_overviewImage.jpg}
\caption{A simple flow of data for the system}
\label{fig:com_systemOverview}
\end{figure}