\label{chap:com_latexoverview}
\section{System overview}

The \appName\ is a server-client system, which involves four different
clients, a Java server and one \term{postgresql} database. The
different kinds of clients are:

\begin{itemize}
\item iOS
\item Android
\item Web
\item Desktop
\end{itemize}

All of these clients use the \term{RESTful} protocol together with
\term{Json} to communicate with the server, sent over a non persistent
\term{HTTP-socket}. How the different requests sent over this socket
is specified in the API which can be found in the appendix or at
$docs.genomizer.apiary.io$.

The server is also divided in different parts, each with a specific
responsibility. These are:

\begin{itemize}
\item Communication - Handles requests and responses
\item Data storage - Handles the database
\item Data transfer - Handles URL and file paths aswell as routing
\item Process - Handles processing of files
\end{itemize}

\refer{fig:com_systemoverview} shows a simple flow diagram which
describes how the client and server communicates. The particular
example shows the data flow when the client process a file.

\begin{figure}[htb]
\addImage{com_systemoverview.jpg}
\caption{A simple flow diagram for the system}
\label{fig:com_systemoverview}
\end{figure}

Every request the client does creates a non persistent connection to
the server. When the server receives a request it checks which kind of
request it is and routes it to either the communication part of the
server or handles it directly. This is done by the data transfer.

If the request is routed to communication a specific command is
created. The command is an object which consists of information from
the \term{RESTful}-header and \term{Json} body sent from the
client. The command is then parsed and sent to different parts of the
server, usually the database first, which returns information from a
\term{SQL query}. Depending on the requests this information can later
be used to, for example processe a file or be sent back to the clients
directly.

The clients are always going to receive a response code after each
request, but in some cases the respond also contains a \term{Json}
body with information which can be shown to the user. This is the case
for requests like $getAnnotations$. The response can also contain
error messages, describing what went wrong when executing the command.

After a client receives the response the connection with the server is
lost until the next request.

There is a special kind of user called system admin. A user with these
priveleges has the rights to add and delete annotations.
