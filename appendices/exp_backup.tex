\section{Introduction}
To backup the genomizer-server rsync and crontab are required. The backup can either be an external storage device (e.g hard drive or USB-storage) on the genomizer-server or a remote backup-server. When using a remote backup-server ssh and a key-pair between the genomizer-server and backup-server is needed. Because rsync synchronizes the genomizer-server and the backup all deleted files on the genomizer-server will be deleted on the backup. To backup from any removal mistakes tar is needed on the backup-server. All backup scripts are located at \emph{resources/backup/} in the genomizer-server project folder.

\section{File backup}
Two scripts are available to perform a backup of the genomizer-server. These are named \emph{local\_file\_backup.sh} and \emph{remote\_file\_backup.sh}.
\emph{local\_file\_backup.sh} will backup to a local path on the server, preferable to an external storage device. \emph{remote\_file\_backup.sh} will backup to a remote backup-server. Both scripts need to be edited to work properly. The following variables must be set:
\begin{itemize}
\item READPATH: Path to folder that you want to backup
\item SAVEPATH: Path to folder where you want to store your backup
\end{itemize}
The script \emph{remote\_file\_backup.sh} also requires the following variables to be set:
\begin{itemize}
\item PORT: ssh port on the backup server
\item USER: User on the backup-server
\item IP: IP address to the backup-server
\end{itemize}
To be able to automate the backup with crontab a ssh key-pair needs to be generated between the genomizer-server and backup-server. 

A script named \emph{backup\_tar.sh} is available to save a compressed tar-file of the genomizer data. This can be used on the backup-server to save a copy of the genomizer data and thus establish some protection against any removal mistakes. The following variables need to be edited in the script:
\begin{itemize}
\item WORKDIR: Path to the parent folder of the genomizer data folder
\item READFOLDER: Name of the genomizer data folder
\item SAVEPATH: Path to folder where you want to store the copy
\item DAYS: Number of days to save the copy. The script will remove any copies older than DAYS days.
\end{itemize}

\section{Database backup}
To backup the database a script named \emph{db\_backup.sh} is available. The following variables need to be edited:
\begin{itemize}
\item DBNAME: Name of the database
\item DBPORT: Port of the databse
\item DBUSER: Username of the database
\item FILENAME: Name of the created backup file
\item SAVEPATH: Path to folder where you want to store the backup file
\end{itemize}
The variable \emph{SAVEPATH} should be the same as the variable \emph{READPATH} in the file backup scripts. Otherwise the database file will not be copied to the backup-server.

To allow the script to access the database a file named 
\begin{verbatim}
.pgpass
\end{verbatim}
needs to be created in the home folder of the user, for example: \\
\emph{/home/username/.pgpass}. \\
This file must contain the following information of the database:
\begin{verbatim}
localhost:PORT:DATABASE:USERNAME:PASSWORD
\end{verbatim}

\section{Crontab}
Crontab is used to automate the backup by scheduling the scripts. Adding or removing scripts in crontab can be made by writing the following in the terminal.
\begin{verbatim}
sudo crontab -e
\end{verbatim}
Here is an example how to add a scheduled script:
\begin{verbatim}
0 0 * * * /path/to/script.sh
\end{verbatim}
Crontab will run the script every midnight all year around.

\section{Restoring}
Restoring the files using a local backup is made by just copying all the backup files to the genomizer data folder. When using a remote backup-server the easiest way is to use rsync from the backup-server to the genomizer-server.

Restoring the database is made by inserting the database backup file into the database. The following command will insert the database file:
\begin{verbatim}
psql dbname < FILENAME.sql
\end{verbatim}
where dbname is the name of the database and FILENAME is the name of the database file.

