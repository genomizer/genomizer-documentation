\section{Introduction}
To backup the genomizer-server \texttt{rsync} and \texttt{crontab} are
required. The backup can either be stored on an external storage
device (e.g hard drive or USB-storage) connected to the
genomizer-server or a remote backup-server.

Using a remote server requires setting up \texttt{ssh} keys for the
genomizer-server to remote backup communication. This can be achieved
using \texttt{ssh-copy-id} or equivalent.

Keep in mind that \texttt{rsync} synchronizes the chosen directory
trees on the genomizer-server and the backup-server. This means that
all deleted files on the genomizer-server will be deleted on the
backup-server.

In addition to the \texttt{rsync} setup, we include a script intended
for use on the backup-server, which packs the synchronized directory
tree into a compressed archive. If added to the backup-servers \texttt{crontab},
this script will automate the process, and also clean up old archives.

All backup scripts are located at \texttt{resources/backup/} in the
genomizer-server project folder.



\section{File backup}

Two scripts are available to perform the directory synchronization
part of the backup of the genomizer-server. These are named
\texttt{local\_file\_backup.sh} and \texttt{remote\_file\_backup.sh},
and correspond to the local or remote storage options mentioned in the
previous section.

\texttt{local\_file\_backup.sh} will backup to a local path on the
genomizer-server. Suggested use is to mount an external storage device
somewhere on the filesystem, and to backup to that.

Conversely, \texttt{remote\_file\_backup.sh} will backup to a remote
backup-server. This is where the \texttt{ssh} keys are required.

Both scripts need to be edited to work properly. The following
variables must be set:

\begin{itemize}
\item READPATH: Path to the directory tree that you wish to backup
\item SAVEPATH: Path to the synchronize to
\end{itemize}

The script \texttt{remote\_file\_backup.sh} also requires the following variables to be set:
\begin{itemize}
\item PORT: The \texttt{ssh} port on the backup-server
\item USER: The user identity to use when connecting to the backup-server
\item IP: IP-address to the backup-server
\end{itemize}

%To be able to automate the backup with \texttt{crontab} a ssh key-pair needs to
%be generated between the genomizer-server and backup-server.

\subsection{Archiving}

The archiving script, \texttt{backup\_tar.sh}, compresses a given
directory tree into a \texttt{gzip}'d tarball. This script should be
invoked at the backup site. This is easily automated by adding the
script to an appropriate \texttt{crontab} at the backup site, and
setting the script to remove old archives at regular intervals.

The following variables need to be edited in the script:
\begin{itemize}
\item WORKDIR: Path to the parent folder of the directory to which the
  genomizer-server is backed up.\footnote{That is, if the data is
    backed up to \texttt{EXAMPLE/PATH/genomizer/data}, then the
    ``WORKDIR'' should be \texttt{EXAMPLE/PATH/genomizer} }
\item READFOLDER: Name of the directory in ``WORKDIR'' to which the
  genomizer-server is backed up.
\item SAVEPATH: Path to the directory in which to store archives 
\item DAYS: Specifies how many days old archives should be stored
  before the script deletes them
\end{itemize}

\section{Database backup}

To backup the database a script called \texttt{db\_backup.sh} is
available. The following variables in the script need to be edited:

\begin{itemize}
\item DBNAME: Name of the database
\item DBPORT: Port of the databse
\item DBUSER: Username for the database
\item FILENAME: Name to give to the created backup file
\item SAVEPATH: Path to the directory in which to store the backup file
\end{itemize}

The variable \texttt{SAVEPATH} should be the same as the variable
\texttt{READPATH} in the file backup scripts. Otherwise the database
file will not be copied to the backup site.

To allow the script to access the database a file called \texttt{.pgpass}
needs to be created in the home directory of the user calling the script.
This file must contain the following information:

\begin{verbatim}
localhost:PORT:DATABASE:USERNAME:PASSWORD
\end{verbatim}

\section{Crontab}

\texttt{Crontab} is the suggested method for automating the invocation
of the backup scripts. Adding system-wide tasks to \texttt{crontab} is
easily achieved using the following command:

\begin{verbatim}
crontab -e
\end{verbatim}

This will open the crontab for the current user, where tasks may then
be defined. Running a background task such as a backup should preferably
be in the root \texttt{crontab} (using \texttt{sudo crontab -e})
or in the \texttt{crontab} of some other ``non-user'' user.

Make sure that the scripts added to \texttt{crontab} are kept in a
secure location, and aren't editable by users without proper access
rights. This is necessary to keep malicious users from adding
arbitrary code to the \texttt{crontab} schedule.
 
Here is a short example of how to add a scheduled script:
\begin{verbatim}
0 0 * * * /path/to/script.sh
\end{verbatim}
The above line tells \texttt{Crontab} to run the script at midnight
every day. For more information on \texttt{crontab} syntax and
behavior, see the \texttt{crontab} manual.

\section{Restoring}

Restoring a backup is much like backing up in reverse. Simply invoke
\texttt{rsync} from the backup site, with the backup directory as
source and the genomizer-server's data directory as target.

To restore from an archive, extract the archive contents to a suitable
location, and then invoke \texttt{rsync} with that directory as source
and the genomizer-server's data directory as target.

%Restoring the files using a local backup is made by just copying all
%the backup files to the genomizer data folder. When using a remote
%backup-server the easiest way is to use \texttt{rsync} from the backup-server
%to the genomizer-server.

Restoring the database is achieved by piping the database dump to
the live database: 
\begin{verbatim}
psql dbname < FILENAME.sql
\end{verbatim}

Equivalently, one could log in to the database and call the dump as a
normal \texttt{sql} script.

Note that in either case, this operation requires the user to have
permission to open and overwrite the database.

