\section{Introduction}
To allow the server to create a mirror of the file system on to a remote backup server, some settings needs to be added. First of make sure that there exists one backup computer 
with internet access and port forwarding to the ssh port (22 by default) running any Linux distribution.

Make sure both computers have the rsync software by typing in:
\begin{verbatim}
man rsync
\end{verbatim}
and make sure there is no error message.

Check if the genomizer server computer have ''crontab'' installed by typing:
\begin{verbatim}
man crontab
\end{verbatim}
and make sure there is no error message. If one of the softwares aren't installed just type:
\begin{verbatim}
sudo apt-get install crontab

OR

sudo apt-get install rsync
\end{verbatim}


\section{File backup}
To synchronize the computer's file systems a simple bash script has to be edited to wanted effect, the script is located at \emph{/var/www/scripts/backup.sh} and looks like this:
\begin{verbatim}
#!/bin/bash
PORT=
USER=
IP=
READPATH=/var/www/data
SAVEPATH=/var/www/data
rsync --ignore-existing --delete --update
         -avze 'ssh -p '$PORT $READPATH $USER@$IP:$SAVEPATH
         
\end{verbatim}
make sure the rsync commad is on one line.
\begin{itemize}
\item Set the PORT variable to the ssh port on the backup server.
\item Set the USER variable to login in the backup server.
\item Set the IP variable to the ip to the backup server.
\end{itemize}
The flags ''-avze'' should be present for the script to work as intended, there are many flags available to change the 
behaviour of the program rsync. Flags to add to create a direct ''mirror'' of the system is:
\begin{itemize}
\item ''--ignore-existing'' - Does not upload files that already exists on the backup server.
\item ''--delete'' - Deletes files on the backup server that does not longer exist on the genomizer server.
\item ''--update'' - Updates files if they have been changed.
\end{itemize}
To see more available flags check:
\url{http://linux.about.com/library/cmd/blcmdl1\_rsync.htm}

To give rsync the right to create folders on the backupserver the /var/www folder needs have its user changed to the user of the system. To do this go to the folder /var/ and type in : 
\begin{verbatim}
sudo chown -R username www/
\end{verbatim}
change the username for the username of the system, now the folders belong to the user and not root.

Finally make the script file runnable by typing:
\begin{verbatim}
sudo chmod -x backup.sh
\end{verbatim}


\section{Database backup}
To synchronize the computer's database a simple bash script has to be edited to wanted effect, the script is located at \emph{/var/www/scripts/sqlback.sh} and looks like this:
\begin{verbatim}
#! /bin/bash
DATE=$(date -I)
DBUSER=
DBNAME=
DBPORT=
SAVEFILE=SqlBackup-$DATE.sql
pg_dump -w -U $DBUSER -h localhost -p $DBPORT $DBNAME > tmp
echo pvt | sudo -S cp tmp /var/www/sqlbackup/$SAVEFILE 
\end{verbatim}
\begin{itemize}
\item Set the DBUSER variable to the username of the database
\item Set the DBNAME variable to the name of the database
\item Set the DBPORT variable to the the port of the databse
\end{itemize}
to allow the script to access the database a file named 
\begin{verbatim}
.pgpass
\end{verbatim}
needs to be created in the home folder of the user for example: \\
\emph{/home/username/.pgpass}. \\
This file should contain the following information:
\begin{verbatim}
localhost:PORT:DATABASE:USERNAME:PASSWORD
\end{verbatim}
where:
\begin{itemize}
\item PORT is changed to the port of the database.
\item DATABASE is the database to be cloned.
\item USERNAME is the username of the postgresql database.
\item PASSWORD is the passdword for the user of the postgresql database.
\end{itemize}

\section{Chrontab}
To enable the server to automatically do syncronizations to the backupserver one line have to be added to a crontab config file.
open the file by typing:
\begin{verbatim}
sudo crontab -e
\end{verbatim}
In the end of the file add a line that looks like this:
\begin{verbatim}
1 0 * * * /var/www/scripts/backup.sh
1 0 * * * /var/www/scripts/sqlback.sh
\end{verbatim}
Explanation of the settings that executes the script:
\begin{enumerate}
\item Setting is the minute (Present: first minute). 
\item Setting is the hour (Present: midnight).
\item Setting is the week (Present: every week in each month)
\item Setting is the month (Present: every month)
\item Setting is the weekday (0=Sunday,1=Monday, Present: every day)
\end{enumerate}
So this backup will run on the first minute of every midnight all year around.
