\label{chap:userstories}

A \term{User Story} is a description of functionality in non technical terms. It describes the wishes of a certain user group and a motivation for why the function is needed.

\section{Implemented user stories}
\begin{figure}[h!]
\userstory{Annotation}{To structure the data files \\ the researchers \\ want to be able to annotate the data files.}
\caption{Annotation user story}
\label{us:annotation}
\end{figure}

\begin{figure}[h!]
\userstory{Single download}{To scrutinize a single data file \\ the researchers \\ want to be able to download a specific file.}
\caption{Download user story}
\label{us:singleDownload}
\end{figure}

\begin{figure}[h!]
\userstory{Single upload}{To store a single data file \\ the researchers \\ want to be able to upload a specific file.}
\caption{Upload user story}
\label{us:singleUpload}
\end{figure}

\begin{figure}[h!]
\userstory{Search for data}{To analyse data \\ the researchers \\ want to be able to search for specific types of data.}
\caption{Search for data user story}
\label{us:searchData}
\end{figure}

\begin{figure}[h!]
\userstory{Batch upload}{To analyse, share and have greater access to data \\ the researchers \\ want to be able to upload multiple files and batch annotate them to a shared location.}
\caption{Batch upload user story}
\label{us:batchUpload}
\end{figure}

\begin{figure}[h!]
\userstory{Raw to profile}{To be able to analyse \\ the researchers \\ want to process raw data to profile data (using bowie and then Philge’s code).}
\caption{Raw to profile user story}
\label{us:rawToProfile}
\end{figure}

\begin{figure}[h!]
\userstory{Delete data}{To save space \\ the researchers \\ want to be able to delete data from the database.}
\caption{Delete data user story}
\label{us:deleteDate}
\end{figure}

\begin{figure}[h!]
\userstory{Change annotation}{To correct and update annotations \\ the researchers \\ want to be able to change data annotations.}
\caption{Change annotation user story}
\label{us:changeAnnotation}
\end{figure}

\begin{figure}[h!]
\userstory{Backup}{To prevent loss of data \\ the researchers \\ want the data to be backed up.}
\caption{Backup user story}
\label{us:backup}
\end{figure}

\begin{figure}[h!]
\userstory{Password protected}{To protect the database from unauthorized use \\ the researchers \\ want the application to be password protected.}
\caption{Password protected user story}
\label{us:passwordProtected}
\end{figure}

\begin{figure}[h!]
\userstory{Add genome release / reference genome}{To be able to annotate the data properly and extract genome reference \\ the researchers \\ want to be able to add genome releases and reference genome.}
\caption{Add genome release / reference genome user story}
\label{us:addGenomeRelease/referenceGenome}
\end{figure}

\begin{figure}[h!]
\userstory{Add chain file}{To be able to convert between genome releases \\ the researchers \\ want to upload chain files (LiftOver).}
\caption{Add chain file user story}
\label{us:addChainFile}
\end{figure}

\begin{figure}[h!]
\userstory{Batch download }{To scrutinize several data files \\ the researchers \\ want to be able to download multiple files at once.}
\caption{Batch download user story}
\label{us:batchDownload}
\end{figure}

\section{Product backlog}

\begin{figure}[h!]
\userstory{File traceabillity}{To be able to access the underlying raw data or profile data \\ the researchers \\ want the raw data files to be traceable from profile files and the profile files to be traceable from the region data (if available) when the files have been generated on the server.}
\caption{File traceabillity user story}
\label{us:fileTraceability}
\end{figure}

\begin{figure}[h!]
\userstory{Convert common file formats}{To get data in a certain convenient file format \\ the researchers \\ want to convert between common file formats (WIG, SGR, GFF3, BED).}
\caption{Convert common file formats user story}
\label{us:convertCommonFileFormats}
\end{figure}

\begin{figure}[h!]
\userstory{Convert genome release}{To easier handle files \\ the researchers \\ want to convert files between genome releases (LiftOver).}
\caption{Convert genome release user story}
\label{us:convertGenomeRelease}
\end{figure}

\begin{figure}[h!]
\userstory{Extract genome reference sequence}{To analyse the reference genome \\ the researchers \\ want extract the reference genome sequence for a given region data.}
\caption{Extract genome reference sequence user story}
\label{us:extractGenomeReferenceSequence}
\end{figure}

\begin{figure}[h!]
\userstory{Advanced batch upload}{To simplify mass upload ~500 files \\ the researchers \\ want to batch annotate files to be uploaded in a spreadsheet.}
\caption{Advanced batch upload user story}
\label{us:advancedBatchUpload}
\end{figure}

\begin{figure}[h!]
\userstory{Profile to region}{To be able to find regions of interest \\ the researchers \\ want to process profile data to region data (Per’s code).}
\caption{Profile to region user story}
\label{us:profileToRegion}
\end{figure}

\begin{figure}[h!]
\userstory{Workspace}{To be able to save work in a convenient way \\ the researchers \\  want to have some sort of workspace view where all kind of results/data can be saved.}
\caption{Workspace user story}
\label{us:workspace}
\end{figure}

\begin{figure}[h!]
\userstory{Unread results}{To avoid missing results \\ the researchers \\  want to see which results are unread.}  \caption{Unread results user story}
\label{us:unreadResults}
\end{figure}

\begin{figure}[h!]
\userstory{Sort search results}{To avoid missing results \\ the researchers \\ want to see which results are unread.}
\caption{Sort search results user story}
\label{us:sortSearchresults}
\end{figure}

\begin{figure}[h!]
\userstory{Preview of file}{To correctly annotatate a file \\ the researchers \\ want to preview a portion of a file.}
\caption{Preview of file user story}
\label{us:previewOfFile}
\end{figure}

\begin{figure}[h!]
\userstory{Work scheduling}{To strategically spread the servers workload over time \\ the researchers \\ want to be able to schedule the processing/analysis of data.}
\caption{Work scheduling user story}
\label{us:workScheduling}
\end{figure}

\begin{figure}[h!]
\userstory{Work queue}{To reduce server load \\ the researchers \\ want to queue time consuming work.}
\caption{Work queue user story}
\label{us:workQueue}
\end{figure}

\begin{figure}[h!]
\userstory{User rights}{To allow invitation of guests (postgraduate students or other researchers etc.) \\ the researchers \\ what to have different users types with different rights.}
\caption{User rights user story}
\label{us:userRights}
\end{figure}

\begin{figure}[h!]
\userstory{Time estimation}{To warn for time consuming jobs \\ the researchers \\ want to have a time estimation for jobs.}
\caption{Time estimation user story}
\label{us:timeEstimation}
\end{figure}

\begin{figure}[h!]
\userstory{Plot overlap analysis}{To see region overlap of genomes \\ the researchers \\ want to plot an overlap analysis (see separate user story).}
\caption{Plot overlap analysis user story}
\label{us:plotOverlapAnalysis}
\end{figure}

\begin{figure}[h!]
\userstory{Plot average regions}{To view data \\ the researchers \\ want to plot average of regions with the profile data.}
\caption{Plot average regions user story}
\label{us:plotAvarageRegions}
\end{figure}

\begin{figure}[h!]
\userstory{IGB Session}{To be able to make IGB analysis \\ the researchers \\ want to retrieve a IGB session file.}
\caption{IGB Session user story}
\label{us:IGBSession}
\end{figure}

\begin{figure}[h!]
\userstory{Combine regions}{To find interesting regions \\ the researches \\ want to select multiple files and combine their regions (union, intersect).}
\caption{Combine regions user story}
\label{us:combineRegions}
\end{figure}

\begin{figure}[h!]
\userstory{Create region subset}{To retrieve certain parts of regions \\ the researchers \\ want to create region subsets using reference points.}
\caption{Create region subset user story}
\label{us:createRegionSubset}
\end{figure}

\begin{figure}[h!]
\userstory{Calculate average of region}{To find the average protein binding value for a region \\ the researchers \\ want to calculate average of regions with the profile data. (Possibly split into a number of bins).}
\caption{Calculate average of region user story}
\label{us:calculateAverageOfRegion}
\end{figure}

\begin{figure}[h!]
\userstory{Overlap analysis}{To conduct overlap analysis \\ the researchers \\ want to divide regions bins, either by value or by order.}
\caption{Overlap analysis user story}
\label{us:overlapAnalysis}
\end{figure}

\begin{figure}[h!]
\userstory{Save analysis results}{To be able to return to previous work \\ the researchers \\ want to save analysis results(in workspace).}
\caption{Save analysis results user story}
\label{us:saveAnalysisResults}
\end{figure}
