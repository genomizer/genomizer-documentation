\label{chap:exp_app_debian}
\section{Introduction}
This document will guide a user through the process to configure the server machine needed for the \appName\ server software. This guide is created while configuring a newly installed machine running Debian 7.5 Wheezy. Other Linux or UNIX operating systems could have other commands to install different software. Some experience with a terminal and the UNIX environment is presumed.

Be sure to have a fully functioning Internet connection to the server machine with the possibility to direct ports to it before continuing.

\section{Installation and Configuration}
The server machine must run a Linux or UNIX operating system. In order to follow this guide in the easiest manner 
possible, use any Debian distribution. 


\subsection{Installation of Debian}
Since Debian is more stable for running server applications than other Linux distributions such as Ubuntu and Linux Mint, it is recommended to use any Debian release.
When partitioning the hard drives make sure to assign at least 2 times the amount of ram as swap section, in our case we used 250GB. This is done to prevent the server from crashing in case the ram RAM is filled out.

The swap is a hardware RAM section that the system can dump to if the RAM is filled.

\subsection{Configure Debian repositories}
To allow the system to download software via terminal a few repositories changes must be made.
To do this open the file:
\begin{verbatim}
sudo nano /etc/apt/sources.list
\end{verbatim}
Then the line
\begin{verbatim}
deb cdrom:[Debian GNU/Linux 7.5.0 _Wheezy_ - 
Official amd64 CD Binary-1 20140426-13:37]/ wheezy main
\end{verbatim}
needs to be commented away from the file to avoid errors when the system tries to fetch software from an non-existent cd rom. Then four repositories should be added to allow installation of software. Add the following lines to the end of the file:
\begin{verbatim}

deb http://ftp.acc.umu.se/debian/ wheezy-updates main
deb-src http://ftp.acc.umu.se/debian/ wheezy-updates main

deb http://ftp.acc.umu.se/debian/ wheezy main
deb-src http://ftp.acc.umu.se/debian/ wheezy main
\end{verbatim}
Try this out by typing:
\begin{verbatim}
sudo apt-get update
\end{verbatim}
and make sure there is no errors.

\subsection{Create a super user}
When Debian is installed there only exists one super user on the computer, and that is ''root''. To give other users on the system root access and super user rights a configuration file must be changed, to open the file you need to login as root temporary (this is not recommended to do for other things).

To login as root, type:
\begin{verbatim}
su
\end{verbatim}
Enter the password for the root, as setup in the installation. Then type the following to open the config file:
\begin{verbatim}
visudo
\end{verbatim}
Then add this line under the line where root is set.
\begin{verbatim}
username ALL=(ALL:ALL) ALL
\end{verbatim}
Where username is the user that will be given super user rights.

\subsection{Locales}
If there is a problem with the locales settings that looks something like this:
\begin{verbatim}

perl: warning: Setting locale failed.
perl: warning: Please check that your locale settings:
	LANGUAGE = "en_GB:en",
	LC_ALL = (unset),
	LC_COLLATE = "sv_SE",
	LC_MEASUREMENT = "sv_SE",
	LC_PAPER = "sv_SE",
	LANG = "C"
    are supported and installed on your system.
perl: warning: Falling back to the standard locale ("C").


\end{verbatim}
Try this fix:
\begin{verbatim}
export LC_ALL=en_GB.UTF-8
sudo /usr/sbin/locale-gen 
sudo dpkg-reconfigure locales
\end{verbatim}
Then reboot the server.

\subsection{Java}
Firstly Java must be installed on the system to be able to run some of the server software. The software requires Java 1.7 or later. 

To install Java JDK open a terminal and enter following command:
\begin{verbatim}
sudo apt-get install openjdk-7-jdk
\end{verbatim}

\subsection{OpenSSH}
To be able to access the server machine remotely OpenSSH must be installed.

Enter the following command to install OpenSSH properly.
\begin{verbatim}
sudo apt-get install openssh-server openssh-client 
openssh-sftp-server
\end{verbatim}

\subsection{Apache2}\label{sec:exp_d_apache2}
In order to handle web requests and file transferring the server will use Apache2. 

To install Apache2 with necessary software, use this command:
\begin{verbatim}
sudo apt-get install apache2 apache2-utils
\end{verbatim}
After installation of Apache2 some configuration is needed. Follow the steps below.
\subsubsection{Configure listening port}\label{sec:exp_d_ports}
The default port for listening is set to 80. If this seeks to be changed follow the steps below.
\begin{enumerate}
	\item Open file with following command: \begin{verbatim}sudo nano /etc/apache2/ports.conf\end{verbatim}
    \item Edit the value in the file after ``Listen'' to the preferred port to use for listening. 
    For example: \begin{verbatim}Listen 80\end{verbatim}
    \item Save and close the file.
    \item Restart the Apache server with: \begin{verbatim}sudo service apache2 graceful\end{verbatim}
\end{enumerate}

\subsubsection{Set document root}\label{sec:exp_d_docroot}
The document root on the Apache server is where the root folder for the server is located. When a user is connecting to the server it will request the content from the root folder of the server.

The \appName\ server uses \textit{/var/www/} as the document root. As default for the Apache server 
the document root is set to \textit{/var/www/html/}.
To change the root folder for the Apache server, do the following steps:
\begin{enumerate}
	\item Open the configuration file for the document root: \begin{verbatim}sudo nano /etc/apache2/sites-enabled/000-default.conf\end{verbatim}
    \item Edit the second string in the line starting with the string ``DocumentRoot'' to the root directory to be used.
    \begin{verbatim}DocumentRoot /var/www/\end{verbatim}
    \item Save and close the configuration file.
    \item Restart the Apache server: \begin{verbatim}sudo service apasche2 graceful\end{verbatim}
\end{enumerate}

After these steps the document root is changed. Please note that in step two the root directory can be set 
to something else. If these steps are followed precisely the document root will be set to \textit{/var/www/}.

\subsubsection{Add system user}\label{sec:exp_d_passw}
To add a system user, you first have to create a new file containing the usernames and their corresponding passwords
by using a terminal and writing:
\begin{verbatim}htpasswd -c /ect/apache2/passwords username\end{verbatim}
Change \texttt{username} to the username wanted for the new user.

This will create a password file in \textit{/etc/apache2/}. 
The path to the password file should not be accessible for the clients. 
Then you will be asked to enter the password for the user:
\begin{verbatim}
New password: mypassword
Re-type new password: mypassword
\end{verbatim}

Instead of \texttt{mypassword}, enter the password wanted for the new user.
When everything is done you will get the message:
\begin{verbatim}Adding password for user username\end{verbatim}
The passwords in the file will be stored encrypted. To add more users, the following command must be used: 
\begin{verbatim}htpasswd /ect/apache2/passwords username\end{verbatim}
If the \texttt{-c} flag is used, a new file will overwrite the old one so all users will be overwritten. For more information, see \cite{exp_apache2user} 

\subsubsection{Setup protected folders for users}\label{sec:exp_d_protected}
The Apache server software have functionality to make protected directories to restrict access to its content. To set this 
up it is necessary to create a user for the Apache server (\textit{see \ref{sec:exp_d_passw}}) to be able to access the files. 

To restrict users from accessing the folders you can make them password protected. To do this, open the file 
\textit{apache2.conf} that is located in \textit{/etc/apache2/}. In the file, new folders can be added. 
These folders will be password protected. To add a directory, new directory tags 
$<$Directory$><$/Directory$>$ must be added among the others.
A step-by-step instruction of how to password protect a folder follows:
\begin{enumerate}
	\item Open the file by the following command:
    \begin{verbatim} sudo nano /etc/apache2/apache2.conf\end{verbatim}
    \item Paste in the following in the file:
\begin{verbatim}
<Directory /path_to_document_root/path_to_restricted_folder/>
    AuthUserFile /etc/apache2/passwords
    AuthName "This is a protected area"
    AuthType Basic
   	Require valid-user
</Directory>
\end{verbatim}
Make sure that the \textit{/path\_to\_document\_root/} is set to the document root set in \ref{sec:exp_d_docroot}. Then \textit{path\_to\_restricted\_folder/} needed to be changed to the actual path to the folder that is to be protected. For more information see \cite{exp_apache2user}.
	\item Make sure your setup is correct.
    \item Save and close the file.
    \item Restart the Apache server to make changes:
    \begin{verbatim}sudo service apache2 graceful\end{verbatim}    
\end{enumerate}

\subsubsection{Setup restricted folders for all users}\label{sec:exp_d_restricted}
It is also possible to make a folder restricted for all users and only accessible through the server machine or the PHP scripts. This is done almost exactly as in \ref{sec:exp_d_protected}. The difference is that you add something else between the directory tags $<$Directory$><$/Directory$>$. Follow step 1 to 5 in \ref{sec:exp_d_protected} but instead paste this to the file at step 2:
\begin{verbatim}
<Directory /path_to_document_root/path_to_restricted_folder/>
    Require all denied
</Directory>
\end{verbatim}


\subsubsection{Setup proxy redirect}
To allow tunneling through the Apache server to the \appName\ server software, a proxy pass has to be set up.
To enable the module for Apache, enter the following command in the terminal:

\begin{verbatim}
sudo a2enmod proxy_http
\end{verbatim}
After the module is loaded a proxy pass has to be set up, the proxy pass works on a url and sends all requests to that url to the proxied address. Start by opening the file:
\begin{verbatim}
sudo nano /etc/apache2/apache2.conf
\end{verbatim}
Scroll down to the end of the file and enter this row (don't forget to modify the url and the proxy to your server setup):
\begin{verbatim}
#Proxypass 
ProxyPass /anyurlyouwant/ http://your.server.address:port/
\end{verbatim}
In this server setup this line looks like this:
\begin{verbatim}
ProxyPass /api/ http://scratchy.cs.umu.se:7000/
\end{verbatim}
This means that all requests sent to \emph{http://scratchy.cs.umu.se:8000/api/Login} will be proxied (tunneled) to
\emph{http://scratchy.cs.umu.se:7000/Login} for example. \textbf{Do not forget to restart the Apache server}:
\begin{verbatim}
sudo service apache2 restart
\end{verbatim}

To allow apache to upload and download files to the system a user called ''www-data'' must be added to the group of the user created for the system. If you don't remember what user you have setup you can write:

\begin{verbatim}
groups
\end{verbatim}
The group name should be present there. now type the following line in the terminal:

\begin{verbatim}
sudo gpasswd -a www-data groupname
\end{verbatim}
where groupname is the user you have setup for the system.

\subsection{Git}
This project uses gitHub for easy sharing of the code between all collaborators. For this to work, the server machine must have git installed to be able to clone the repositories with code. 

This document will not specify how to use git, instead please read the two guides \cite{exp_gitguide} and \cite{exp_gitguide2}

To install git on the server machine, enter the following line in the terminal:
\begin{verbatim}sudo apt-get install git\end{verbatim}
For easy access to gitHub, SSH-keys can be added to easily execute push and pull of repositories. 
See gitHub's guide \cite{exp_sshguide}.

\subsection{PHP5}
The Apache server uses PHP scripts to upload and download files. Therefore it is necessary to install PHP5 on the server machine. To install PHP5, just enter the following command in a terminal:
\begin{verbatim}sudo apt-get install php5-curl\end{verbatim}

The following file needs to be configurated \begin{verbatim}/etc/php5/apache2/php.ini\end{verbatim} with values:

\begin{verbatim}/etc/php5/apache2filter/php.ini\end{verbatim}

\begin{verbatim}max_execution_time to 0\end{verbatim}
\begin{verbatim}max_input_time to -1\end{verbatim}
\begin{verbatim}upload_max_filesize to 0\end{verbatim}
\begin{verbatim}post_max_size to 0\end{verbatim}

\subsection{SRA Toolkit}
One of the PHP scripts will need the application SRA Toolkit installed. To install this application, enter the following in the terminal:
\begin{verbatim}sudo apt-get install sra-toolkit\end{verbatim}
This application is used to convert .sra files to .fastq files. To manually use SRA Toolkit enter the following in the terminal:
\begin{verbatim}fastq-dump /var/www/test/SRR869740.sra\end{verbatim}
This will open the SRR869740.sra file to a SRR869740.fastq file in the same directory of the original .sra file.

\subsection{PostgreSQL}
For the server machine, PostgreSQL is required for the server to work as intended. To install PostgreSQL, enter the following command (note: the version 9.3 may vary):
\begin{verbatim}
sudo apt-get install postgresql-9.1 postgresql-client-9.1
 postgresql-contrib-9.1
\end{verbatim}

An admin account needs to be set up for the database, to do this follow these instructions:

\begin{enumerate}
\item Login to the PostgreSQL server by typing \begin{verbatim}  sudo -u postgres psql \end{verbatim}
where \texttt{postgres} is the default user for the database
\item Enter the following command while inside psql to set up a password for the user postgres:
\begin{verbatim}
ALTER ROLE postgres WITH ENCRYPTED PASSWORD 'password';
\end{verbatim}
and change \texttt{password} to whatever password is wanted.
\end{enumerate}

To grant access to the database from non-local machines, the following file must be changed (note: the version 9.3 may vary):
\begin{verbatim}
sudo nano /etc/postgresql/9.3/main/postgresql.conf
\end{verbatim}
Find the segment \emph{CONNECTIONS AND AUTHENTICATION} in the top part of the file and change the lines ''listen\_addresses'' and ''port'':

\begin{verbatim}
#----------------------------------------------------------
# CONNECTIONS AND AUTHENTICATION
#----------------------------------------------------------

# - Connection Settings -

listen_addresses = '*'  # what IP address(es) to listen on;
                        # comma-separated list of addresses;
                        # defaults to 'localhost';
                        # use '*' for all
                        # (change requires restart)
port = 6000             # (change requires restart)

\end{verbatim}

Port of the server can be changed to whatever is wished. Now the access needs to be changed. To do this add the following lines to the file (note: the version 9.3 may vary): 
\begin{verbatim}
sudo nano /etc/postgresql/9.3/main/pg_hba.conf
\end{verbatim}
Make sure that there exists 2 lines that look like the following (change existing lines or add new ones):
\begin{verbatim}
# "local" is for Unix domain socket connections only
local   all             all                                  md5
# IPv4 local connections:
host    all             all             127.0.0.1/32         md5
\end{verbatim}
Restart PostgreSQL by typing:
\begin{verbatim}
sudo service postgresql restart
\end{verbatim}


\subsubsection{Clone database}
If there exists an old database that is wished to be migrated to the new database the following command can be executed on the machine where the database is presently:

\begin{verbatim}
sudo pg_dump -U dbUserName -d dbName -h localhost -p
dbPort > backupfile.sql
\end{verbatim}
\begin{enumerate}
\item Change \verb+dbUserName+ to the username you have setup for PostgreSQL
\item Change \verb+dbName+ to the name of the database that is wished to be migrated
\item Change \verb+dbPort+ to the PostgreSQL port which it is setup to listen to
\item Change \verb+backupfile.sql+ to whatever filename is wished
\end{enumerate}

This creates a backup SQL file. Now transfer the file to the server where the database is migrated to and type in the following command to inject it into the database:

\begin{verbatim}
psql -U dbUserName -h localhost -d dbName -p dbPort < backupfile.sql
\end{verbatim}
\begin{enumerate}
\item Change \verb+dbUserName+ to the username you have setup for PostgreSQL
\item Change \verb+dbName+ to the name of the database that is wished to be migrated to
\item Change \verb+dbPort+ to the PostgreSQL port which it is setup to listen to
\item Change \verb+backupfile.sql+ to whatever it is named
\end{enumerate}
Restart PostgreSQL by typing:
\begin{verbatim}
sudo service postgresql restart
\end{verbatim}




\subsection{PgAdmin}\label{sec:exp_d_pgadmin}
PgAdmin is a software which provides a graphical interface towards the PostgresQL server and can be installed with following command:
\begin{verbatim}
sudo apt-get install pgadmin3
\end{verbatim}





\subsection{PhpPgAdmin}
PhpPgAdmin (\refer{fig:exp_d_phppgadminpic}) is a user friendly web interface that connects to the server PostgreSQL database. 
This is recommended to be installed if you are not very comfortable working with the database using the terminal 
interface or wish to only configure the database on the local server machine using PgAdmin (\ref{sec:exp_d_pgadmin}). 


\begin{figure}[htb]
\centering
\addImage{exp_phppgadmin.jpg}
\caption{PhpPgAdmin web interface}
\label{fig:exp_d_phppgadminpic}
\end{figure}

\subsubsection{Setup PhpPgAdmin}

Then install the required software by typing in the following command in the terminal:

\begin{verbatim}
sudo apt-get install phppgadmin
\end{verbatim}

This needs to be included by the Apache2 software, which is done by editing the file:

\begin{verbatim}
sudo nano /etc/apache2/apache2.conf
\end{verbatim}

and adding this line to the end of the file after the other includes:

\begin{verbatim}
#Include Phppgadmin
Include /etc/apache2/conf.d/phppgadmin
\end{verbatim}

When that is done, we need to change the access settings to the phppgadmin via the Apache software, this is done by changing the file:

\begin{verbatim}
sudo nano /etc/apache2/conf.d/phppgadmin
\end{verbatim}
In the top part of the file a section is displayed as below:

\begin{verbatim}
order deny,allow
deny from all
allow from 127.0.0.0/255.0.0.0 ::1/128
#allow from all
\end{verbatim}
Change this section so that it looks like this:
\begin{verbatim}
order deny,allow
#deny from all
allow from 127.0.0.0/255.0.0.0 ::1/128
allow from all
\end{verbatim}

Now an account must be set up with the PhpPgAdmin. Make sure you have the \emph{htpasswd} software installed 
(comes with Apache2-utils). Then to set an account, enter the following command in the terminal:

\begin{verbatim}
sudo htpasswd -c /etc/phppgadmin/.htpasswd <username>
\end{verbatim}
Change the username to what you want the user to be called.
After that a prompt will be shown to enter a password, enter the password twice and then the account is setup.

Now the Apache server needs to be told where to look for the users. This is done by editing the file:


\begin{verbatim}
sudo nano /etc/apache2/sites-enabled/000-default.conf
\end{verbatim}
Add this to the end of the file:

\begin{verbatim}
<Directory "/usr/share/phppgadmin">
        AuthUserFile /etc/phppgadmin/.htpasswd
        AuthName "Restricted Area"
        AuthType Basic
        require valid-user
</Directory>

\end{verbatim}

Now PhpPgAdmin needs to be told which port to connect to the PostgreSQL on (se configurations of the PostgreSQL server). %TODO ref
To do that changes needs to be made to the file:
\begin{verbatim}
sudo nano /etc/phppgadmin/config.inc.php
\end{verbatim}
Then change the following post to what corresponds to your server setup:
\begin{verbatim}
// Database port on server (5432 is the PostgreSQL default)
    $conf['servers'][0]['port'] = 6000; //PostgreSQL port here 
    .
    .
    .

// passworded local connections.
    	$conf['extra_login_security'] = false; //True as standard

\end{verbatim}




Restart Apache and phppgadmin by typing:
\begin{verbatim}
sudo service apache2 restart
sudo service phppgadmin restart
\end{verbatim}
