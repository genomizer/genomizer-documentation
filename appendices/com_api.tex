\label{chap:com_api}

The server relies on REST to handle communication between the server and the clients. There are a set of valid requests that can be made, this section encompasses all of them. They are divided into respective categories, those being Connection, Experiment, File, File conversion, Search, User, Admin, Processing, Annotation handling, Genome release handling, GEO and File upload/download.

All requests are briefly explained and example requests and responds are given.

\section*{Connection}
Requests used to login/logout to and from the server.

\subsection*{POST /login}

Logs in as a user.

\subsubsection*{Request:}

\begin{verbatim}
POST /login HTTP/1.1
Content-Type: application/json
{
 "username": "uname", 
 "password": "pw"
}
\end{verbatim}

\subsubsection*{Response:}

\begin{verbatim}
200 (OK)
Content-Type: application/json
{
 "token": "token",
 "role": "role"
}
\end{verbatim}

\subsection*{DELETE /login}

Logs out as a user.

\subsubsection*{Request:}

\begin{verbatim}
DELETE /login HTTP/1.1
authorization: token
\end{verbatim}
\subsubsection*{Response:}

\begin{verbatim}
200 (OK)
\end{verbatim}

\section*{Experiment}
Requests used to add, retrieve, update and delete experiments.

\subsection*{POST /experiment}

Creates a new experiment.

\subsubsection*{Request:}
\begin{verbatim}
POST /experiment HTTP/1.1
Content-Type: application/json
Authorization: token
{
    "name": "experimentId",
    "annotations": 
                 [
                 {
                  "name": "pubmedId",
                  "value": "abc123"
                 }, 
                 {
                  "name": "type",
                  "value": "raw"
                 },
                 {
                  "name": "specie",
                  "value": "human"
                 },
                 {
                  "name": "genome release",
                  "value": "v.123"
                 },
                 {
                  "name": "cell line",
                  "value": "yes"
                 },
                 {
                  "name": "development stage",
                  "value": "larva"
                 },
                 {
                  "name": "sex",
                  "value": "male"
                 },
                 {
                  "name": "tissue",
                  "value": "eye"
                 }
                 ]

}
\end{verbatim}

\subsubsection*{Response:}
\begin{verbatim}
200 (OK)
\end{verbatim}	

\subsection*{GET /experiment/<experiment-id>}

Retrieves a specific experiment.

\subsubsection*{Request:}
\begin{verbatim}
GET /experiment/<experiment-id> HTTP/1.1
Authorization: token
\end{verbatim}

\subsubsection*{Response:}
\begin{verbatim}
200 (OK)
Content-Type: application/json
Authorization: token
{
    "name": "experimentId",
    "files": [
            { 
             "id": "id",
             "path": "path",
             "url": "url",
             "type": "type",
             "filename": "filename",
             "date": "date",
             "metaData": "metaData",
             "author": "author",
             "uploader": "uploader",
             "expId": "expId",
             "grVersion": "realseNr",
             "fileSize": "fileSize"
            },
            { 
             "id": "id",
             "path": "path",
             "url": "url",
             "type": "type",
             "filename": "filename",
             "date": "date",
             "metaData": "metaData",
             "author": "author",
             "uploader": "uploader",
             "expId": "expId",
             "grVersion": "realseNr",
             "fileSize": "fileSize"
            }
             ],
    "annotations": 
                 [
                 {
                  "name": "pubmedId",
                  "value": "abc123"
                 }, 
                 {
                  "name": "type",
                  "value": "raw"
                 },
                 {
                  "name": "specie",
                  "value": "human"
                 },
                 {
                  "name": "genome release",
                  "value": "v.123"
                 },
                 {
                  "name": "cell line",
                  "value": "yes"
                 },
                 {
                  "name": "development stage",
                  "value": "larva"
                 },
                 {
                  "name": "sex",
                  "value": "male"
                 },
                 {
                  "name": "tissue",
                  "value": "eye"
                 }
                 ]

}
\end{verbatim}

\subsection*{PUT /experiment/<experiment-id>}

Edits an exisiting experiment.

\subsubsection*{Request:}
\begin{verbatim}
PUT /experiment/<experiment-id> HTTP/1.1
Content-Type: application/json
Authorization: token
{
    "name": "experimentId",
    "annotations": 
                 [
                 {
                  "name": "pubmedId",
                  "value": "abc123"
                 }, 
                 {
                  "name": "type",
                  "value": "raw"
                 },
                 {
                  "name": "specie",
                  "value": "human"
                 },
                 {
                  "name": "genome release",
                  "value": "v.123"
                 },
                 {
                  "name": "cell line",
                  "value": "yes"
                 },
                 {
                  "name": "development stage",
                  "value": "larva"
                 },
                 {
                  "name": "sex",
                  "value": "male"
                 },
                 {
                  "name": "tissue",
                  "value": "eye"
                 }
                 ]
    }
\end{verbatim}

\subsubsection*{Response:}
\begin{verbatim}
200 (OK)
\end{verbatim}

\subsection*{DELETE /experiment/<experiment-id>}

Deletes an existing experiment.

\subsubsection*{Request:}
\begin{verbatim}
DELETE /experiment/<experiment-id> HTTP/1.1
Authorization: token
\end{verbatim}

\subsubsection*{Response:}
\begin{verbatim}
200 (OK)
\end{verbatim}

\section*{File}

Requests ued to add, retrieve (file information), update and delete files.

\subsection*{POST /file}

Adds a file to an experiment.

\subsubsection*{Request:}
\begin{verbatim}
POST /file HTTP/1.1
Content-Type: application/json
Authorization: token
{ 
 "experimentID": "id",
 "fileName": "name",
 "type": "raw",
 "metaData": "metameta",
 "author": "name",
 "grVersion": "releaseNr"
 "checkSumMD5": "checksum"
}
\end{verbatim}

\subsubsection*{Response:}
\begin{verbatim}
200 (OK)
\end{verbatim}

\subsection*{GET /file/<file-id>}

Retrieves file information about a specific file.

\subsubsection*{Request:}
\begin{verbatim}
GET /file/<file-id> HTTP/1.1
Authorization: token
\end{verbatim}

\subsubsection*{Response:}
\begin{verbatim}
200 (OK)
Content-Type: application/json
{ 
 "id": "id",
 "path": "path",
 "url": "url",
 "type": "type",
 "filename": "filename",
 "date": "date",
 "metaData": "metaData",
 "author": "author",
 "uploader": "uploader",
 "expId": "expId",
 "grVersion": "realseNr",
 "fileSize": "fileSize",
 "checkSumMD5": "checkSumMD5"
}
\end{verbatim}

\subsection*{PUT /file/<file-id>}

Edits file information for a specific file.

\subsubsection*{Request:}
\begin{verbatim}
PUT /file/<file-id> HTTP/1.1
Content-Type: application/json
Authorization: token
{ 
 "experimentID": "id",
 "filename": "name",
 "type": "raw",
 "metaData": "metameta",
 "author": "name",
 "grVersion": "releaseNr"
 "checkSumMD5": "checksum"
}
\end{verbatim}

\subsubsection*{Response:}
\begin{verbatim}
200 (OK)
\end{verbatim}

\subsection*{DELETE /file/<file-id>}

Deletes a specific file.

\subsubsection*{Request:}
\begin{verbatim}
DELETE /file/<file-id> HTTP/1.1
Authorization: token
\end{verbatim}

\subsubsection*{Response:}
\begin{verbatim}
200 (OK)
\end{verbatim}

\section*{File conversion}

Requests used to handle file conversion.

\subsection*{PUT /convertfile}

Converts a file to a given format.

\subsubsection*{Request:}
\begin{verbatim}
PUT /convertfile HTTP/1.1
Content-Type: application/json
Authorization: token
{ 
 "fileid": "id", 
 "toformat": "format"
}
\end{verbatim}

\subsubsection*{Response:} 
\begin{verbatim}
200 (OK)
Content-Type: application/json
    {
     "id": "id",
     "path": "path",
     "url": "url",
     "type": "type",
     "filename": "filename",
     "date": "date",
     "metaData": "metaData",
     "author": "author",
     "uploader": "uploader",
     "expId": "expId",
     "grVersion": "realseNr",
     "fileSize": "fileSize",
     "checkSumMD5": "checkSumMD5"
    }
\end{verbatim}

\section*{Search}

Requests used to handle user information.

\subsection*{PUT /user}

Edits the user information.

\subsubsection*{Request:}
\begin{verbatim}
PUT /user HTTP/1.1
Content-Type: application/json
Authorization: token
{
 "oldPassword": "oldPw"
 "newPassword": "newPw",
 "name": "John Johnson",
 "email": "john@mail.com"
}
\end{verbatim}


\subsubsection*{Response:}
\begin{verbatim}
200 (OK)
\end{verbatim}

\subsection*{GET /user/<username>}

\subsubsection*{Request:}
\begin{verbatim}
GET /user/<username> HTTP/1.1
Authorization: token
\end{verbatim}

\subsubsection*{Response:}
\begin{verbatim}
200 (OK)
Content-Type: application/json
    {
     "username": "myusername"
     "privileges": "ADMIN",
     "name": "John Johnson",
     "email": "john@mail.com"
    }
\end{verbatim}

\section*{Search}

Requests used to handle searching for experiments.

\subsection*{GET /search/?annotations=<pubmedStyleQuery>}

Searches for an experiment using a pubmed query.

\subsubsection*{Request:}
\begin{verbatim}
GET /search/?annotations=<pubmedStyleQuery> HTTP/1.1
Authorization: token
\end{verbatim}

\subsubsection*{Response:}
\begin{verbatim}
200 (OK)
Content-Type: application/json
[
   {
        "name": "experimentId",
        "files": [
                { 
                 "id": "id",
                 "path": "path",
                 "url": "url",
                 "type": "type",
                 "filename": "filename",
                 "date": "date",
                 "metaData": "metaData",
                 "author": "author",
                 "uploader": "uploader",
                 "expId": "expId",
                 "grVersion": "realseNr",
                 "fileSize": "fileSize"
                },
                { 
                 "id": "id",
                 "path": "path",
                 "url": "url",
                 "type": "type",
                 "filename": "filename",
                 "date": "date",
                 "metaData": "metaData",
                 "author": "author",
                 "uploader": "uploader",
                 "expId": "expId",
                 "grVersion": "realseNr",
                 "fileSize": "fileSize"
                }
                 ],
        "annotations": 
                     [
                     {
                      "name": "pubmedId",
                      "value": "abc123"
                     }, 
                     {
                      "name": "type",
                      "value": "raw"
                     },
                     {
                      "name": "specie",
                      "value": "human"
                     },
                     {
                      "name": "genome release",
                      "value": "v.123"
                     },
                     {
                      "name": "cell line",
                      "value": "yes"
                     },
                     {
                      "name": "development stage",
                      "value": "larva"
                     },
                     {
                      "name": "sex",
                      "value": "male"
                     },
                     {
                      "name": "tissue",
                      "value": "eye"
                     }
                     ]

    }
]
\end{verbatim}

\section*{Processing}

Requests used to process experiment data.

\subsection*{GET /process}

Gets the statuses of all processes.

\subsubsection*{Request:}
\begin{verbatim}
GET /process HTTP/1.1
Authorization: token
\end{verbatim}

\subsubsection*{Response:}
\begin{verbatim}
200 (OK)
Content-Type: application/json
[
  {
    "experimentName": "Exp1",
    "PID": "123",
    "status": "Finished",
    "outputFiles": [
      "file1",
      "file2"
    ],
    "author": "yuri",
    "timeAdded": 1400245668744,
    "timeStarted": 1400245668756,
    "timeFinished": 1400245669756
  },
  {
    "experimentName": "Exp2",
    "PID": "124",
    "status": "Finished",
    "outputFiles": [
      "file1",
      "file2"
    ],
    "author": "janne",
    "timeAdded": 1400245668746,
    "timeStarted": 1400245669756,
    "timeFinished": 1400245670756
  },
  {
    "experimentName": "Exp43",
    "PID": "125",
    "status": "Crashed",
    "outputFiles": [
      "file1",
      "file2"
    ],
    "author": "philge",
    "timeAdded": 1400245668748,
    "timeStarted": 1400245670756,
    "timeFinished": 1400245671757
  },
  {
    "experimentName": "Exp234",
    "PID": "126",
    "status": "Started",
    "outputFiles": [
      "file1",
      "file2"
    ],
    "author": "per",
    "timeAdded": 1400245668753,
    "timeStarted": 1400245671757,
    "timeFinished": 0
  },
  {
    "experimentName": "Exp6",
    "PID": "127",
    "status": "Waiting",
    "outputFiles": [],
    "author": "yuri",
    "timeAdded": 1400245668755,
    "timeStarted": 0,
    "timeFinished": 0
  }
]
\end{verbatim}

\subsection*{DELETE /process}

Cancels a running process.

\subsubsection*{Request:}
\begin{verbatim}
DELETE /process HTTP/1.1
Content-Type: application/json
Authorization: token
{
 "PID": "PID"
}
\end{verbatim}

\subsubsection*{Response:}
\begin{verbatim}
200 (OK)
\end{verbatim}

\subsection*{PUT /process/rawtoprofile}

Starts a raw to profile processing.

\subsubsection*{Request:}
\begin{verbatim}
PUT /process/rawtoprofile HTTP/1.1
Content-Type: application/json
Authorization: token
{
"expid": "expID",
"parameters": 
            [
             "Bowtie parameters",
             "",
             "y",
             "",
             "10 1 5 0 0",
             "y 10",
             "single 4 0",
             "150 1 7 0 0"
            ],
"metadata": "metaData",
"genomeVersion": "hg38"
}
\end{verbatim}

\subsubsection*{Response:}
\begin{verbatim}
200 (OK)
\end{verbatim}

\subsection*{PUT /process/processCommands}

Lists files and processes to run on them.

\subsubsection*{Request:}
\begin{verbatim}
PUT /process/processCommands HTTP/1.1
Content-Type: application/json
Authorization: token
{
    "expId": ...,
    "processCommands": [
        {
            "type": "bowtie",
            "files": [
                {
                    "infile": ...,
                    "outfile": ...,
                    "params": ...,
                    "genomeVersion": ...,
                    "keepSam": true/false
                },
                ...
            ]   
        },
        {
            "type": "ratio",
            ...
            (not supported)
        },
        ...
    ]
}
\end{verbatim}

\subsubsection*{Response:}
\begin{verbatim}
200 (OK)
\end{verbatim}

\section*{Annotation handling}

Requests used to handle annotations.

\subsection*{GET /annotation}

Gets all existing annotations.

\subsubsection*{Request:}
\begin{verbatim}
GET /annotation HTTP/1.1
Authorization: token
\end{verbatim}

\subsubsection*{Response:}
\begin{verbatim}
200 (OK)
Content-Type: application/json
[
 {
  "name": "pubmedId",
  "values": ["freetext"],
  "forced": true
 }, 
 {
  "name": "type",
  "values": ["freetext"],
  "forced": true
 },
 {
  "name": "specie",
  "values": ["fly", "human", "rat"],
  "forced": true
 },
 {
  "name": "genome release",
  "values": ["freetext"],
  "forced": true
 },
 {
  "name": "cell line",
  "values": ["yes", "no"],
  "forced": true
 },
 {
  "name": "development stage",
  "values": ["larva", "larvae"],
  "forced": true
 },
 {
  "name": "sex",
  "values": ["male", "female", "unknown"],
  "forced": true
 },
 {
  "name": "tissue",
  "values": ["eye", "leg"],
  "forced": false
 }
]
\end{verbatim}

\subsection*{POST /annotation/field}

Adds an annotation field.

\subsubsection*{Request:}
\begin{verbatim}
POST /annotation/field HTTP/1.1
Content-Type: application/json
Authorization: token
{ 
 "name": "species",
 "type": [
          "fly",
          "rat",  
          "human" 
         ],
 "default": "human", 
 "forced": false
}
\end{verbatim}

\subsubsection*{Response:}
\begin{verbatim}
200 (OK)
\end{verbatim}

\subsection*{PUT /annotation/field}

Renames an annotation field.

\subsubsection{Request:}
\begin{verbatim}
PUT /annotation/field HTTP/1.1
Content-Type: application/json
Authorization: token
{
 "oldName": "species",
 "newName": "mouse"
}
\end{verbatim}

\subsubsection*{Response:}
\begin{verbatim}
200 (OK)
\end{verbatim}

\subsection*{DELETE /annotation/field/<field-name>}

Deletes an annotation field.

\subsubsection*{Request:}
\begin{verbatim}
DELETE /annotation/field/<field-name> HTTP/1.1
Authorization: token
\end{verbatim}

\subsubsection*{Response:}
\begin{verbatim}
200 (OK)
\end{verbatim}

\subsection*{POST /annotation/value}

Adds an annotation value.

\subsubsection*{Request:}
\begin{verbatim}
POST /annotation/value HTTP/1.1
Content-Type: application/json
Authorization: token
{
 "name": "species", 
 "value": "mouse"
}
\end{verbatim}

\subsubsection*{Response:}
\begin{verbatim}
200 (OK)
\end{verbatim}

\subsection*{PUT /annotation/value}

Renames an annotation value.

\subsubsection*{Request:}
\begin{verbatim}
PUT /annotation/value HTTP/1.1
Content-Type: application/json
Authorization: token
{
 "name": "species",
 "oldValue": "mouse",
 "newValue": "rat"
}
\end{verbatim}

\subsubsection*{Response:}
\begin{verbatim}
200 (OK)
\end{verbatim}

\subsection*{DELETE /annotation/value/<field-name>/<value-name>}

\subsubsection*{Request:}
\begin{verbatim}
DELETE /annotation/value/<field-name>/<value-name> HTTP/1.1
Authorization: token
\end{verbatim}

\subsubsection*{Response:}
\begin{verbatim}
200 (OK)
\end{verbatim}

\section*{Genome release handling}

Requests used to add, get, and delete genome releases.

\subsection*{GET /genomeRelease}

Retrieves all genome releases.

\subsubsection*{Request:}
\begin{verbatim}
GET /genomeRelease HTTP/1.1
Authorization: token
\end{verbatim}

\subsubsection*{Response:}
\begin{verbatim}
200 (OK)
Content-Type: application/json
[
 {
  "genomeVersion": "hy17",
  "species": "fly",
  "folderPath": "pathToVersion",
  "files": ["filename1", "filename2", "filename3"]
 }, 
 {
  "genomeVersion": "u12b",
  "species": "human",
  "folderPath": "pathToVersion",
  "files": ["filename1", "filename2"]
 },
 {
  "genomeVersion": "wk1m",
  "species": "human",
  "folderPath": "pathToVersion",
  "files": ["filename1", "filename2"]
 },
 {
  "genomeVersion": "fg2b",
  "species": "rat",
  "folderPath": "pathToVersion",
  "files": ["filename1", "filename2"]
 },
 {
  "genomeVersion": "abc1",
  "species": "rat",
  "folderPath": "pathToVersion",
  "files": ["filename1", "filename2"]
 }
]
\end{verbatim}

\subsection*{POST /genomeRelease}

Adds a genome release.

\subsubsection*{Request:}
\begin{verbatim}
POST /genomeRelease HTTP/1.1
Content-Type: application/json
Authorization: token
{
 "genomeVersion": "hx16",
 "specie": "human", 
 "files":
        [
         "nameOfFile1",
         "nameOfFile2",
         "nameOfFile3"
        ],
 "checkSumMD5": 
              [
               "checkSum1",
               "checkSum2",
               "checkSum3"
              ]
}
\end{verbatim}

\subsubsection*{Response:}
\begin{verbatim}
200 (OK)
Content-Type: application/json
[
 {
  "URLupload": "url1"
 },
 {
  "URLupload": "url2"
 },
 {
  "URLupload": "url3"
 }
]
\end{verbatim}

\subsection*{GET /genomeRelease/<species>}

Retrieves the genome releases for a specific species.

\subsubsection*{Request:}
\begin{verbatim}
GET /genomeRelease/<species> HTTP/1.1
Authorization: token
\end{verbatim}

\subsubsection*{Response:}
\begin{verbatim}
200 (OK)
Content-Type: application/json
 [
  {
   "genomeVersion": "fg2b",
   "species": "rat",
   "folderPath": "pathToVersion",
   "files": ["filename1", "filename2"]
  },
  {
   "genomeVersion": "abc1",
   "species": "rat",
   "folderPath": "pathToVersion",
   "files": ["filename1", "filename2"]
  }
 ]
\end{verbatim}

\subsection*{DELETE /genomeRelease/<species>/<genomeVersion>}

Deletes a genome release.

\subsubsection*{Request:}
\begin{verbatim}
DELETE /genomeRelease/<species>/<genomeVersion> HTTP/1.1
Authorization: token
\end{verbatim}

\subsubsection*{Response:}
\begin{verbatim}
200 (OK)
\end{verbatim}

\section*{File upload/download}

Requests used to upload and download files.

\subsection*{POST /upload?path=<path-on-server>}

Uploads a file to the server.

\subsubsection*{Request:}
\begin{verbatim}
POST /upload?path=/path/on/server/foo.bar HTTP/1.1
Authorization: token
Content-Type: multipart/form-data; boundary=---------------------------9051914041544843365972754266
Content-Length: 12345
---------------------------9051914041544843365972754266
Content-Disposition: form-data; name="file1"; filename="foo.txt"
Content-Type: text/plain

Contents of foo.txt.
\end{verbatim}

\subsubsection*{Response:}
\begin{verbatim}
201 (Created)
\end{verbatim}

\subsection*{GET /download?path=<path-on-server>}

Downloads a file from the server.

\subsubsection*{Request:}
\begin{verbatim}
GET /download?path=/path/on/server/foo.bar HTTP/1.1
Authorization: token
\end{verbatim}

\subsubsection*{Response:}
\begin{verbatim}
200 (OK)
Content-Description: File Transfer
Content-Type: application/octet-stream
Content-Disposition: attachment; filename=foo.bar
Expires: 0
Cache-Control: must-revalidate
Pragma: public
Header: foo
[contents of foo.bar]
\end{verbatim}
