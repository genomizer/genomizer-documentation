The \textit{genomizer-server-tester} is a separate program from the rest of the server system that uses parts from
the desktop clients code. The purpose of the program is to in sequence execute all possible commands that the server should be able to handle.

The program was given a repository under the main \textit{genomizer} repository. The repository is named \textit{genomizer-server-tester}. 

\section{User guide}
The repository contains the a build script that creates a jar of the program with the help of \textit{Ant} as follows:
\begin{verbatim}
ant jar
\end{verbatim}

The program then is ran with the help of one argument which is the address of the server.
\begin{verbatim}
java -jar genomizer-server-tester.jar <address>
\end{verbatim}

The address supplied has to be \texttt{https://} followed by a \texttt{/api} to redirect the program right.

The program then runs a mulitple of pre-defined tests and prints the result on standard output.

\begin{example}
	\begin{verbatim}
		...
		...
		...
		--------------- USER ----------------------
		TEST: POST ADMIN USER 		STATUS: SUCCESS
		TEST: DELETE ADMIN USER     STATUS: FAILED
		.
		.
		.
		-------------------------------------------
		Total tests run: 113
		Successfull tests: 109
		Failed tests: 4
		-------------------------------------------
		Failed:
		DELETE ADMIN USER
		-------------------------------------------
		
	\end{verbatim}
\end{example}


\section{Program structure}
The program uses the desktop style for upload, requests and connection so the tests ran will give some feedback to desktop code as well.

The tests is built with the help of two classes \class{SuperTestClass} and \class{TestCollection}.
\class{TestCollection} contains a list of \class{SuperTestClass} and has an execute where it runs tests.

The \class{SuperTestClass} insures that for each tests constructed the test promises to have at least three things, some have a forth.

\begin{enumerate}
	\item The test has a identifier string.
	\item The test has a expected result.
	\item The test has a execute method.
	\item Some test has a expected result string to match the result received from the server.
\end{enumerate}

\subsection*{Example}
First each test has to have a collection.
\begin{verbatim}
public class CollectionExample extends TestCollection {
	private String testName;	
	
	public CollectionExample () {
		super()
		this.testName = "test";
		
		super.commandList.add(new exampleTest("IDENT", true));
	}
	
	@Override
	public boolean execute() {
		System.out.println("------------ TEST COLLECTION -----------");
		boolean bigResult = true;
		
		for(SuperTestCommand stc: super.commandList) {
			stc.execute();
			
			runTests++;
			
			boolean succeeded = stc.finalResult == stc.expectedResult;

            if (succeeded){
                succeededTests++;
            } else {
                failedTests++;
                nameOfFailedTests.add(stc.ident);
                bigResult = false;
            }

            Debug.log(stc.toString(), succeeded);
            try {
                Thread.sleep(delay);
            } catch (InterruptedException e) {
                e.printStackTrace();
            }			
			
		}
		return bigResult;
	}
}
\end{verbatim}

Then construct a test.

\begin{verbatim}
public class ExampleTest extends SuperTestCommand {
	
	public ExampleTest(String ident, boolean expectedResult) {
		super(ident, expectedResult);
	}
	
	@Override
	public void execute() {
		try {
			CommandTester.conn.sendRequest(new SearchRequest(""),
                    CommandTester.token, Constants.JSON);

            if (CommandTester.conn.getResponseCode() == 200){
                super.finalResult = true;
            }
        } catch (RequestException e) {
            if (super.expectedResult) ErrorLogger.log(e);
        }
   }
   
}
\end{verbatim}
And finally add the test to the \texttt{CommandTester.java}.

\begin{verbatim}
	CollectionExample ex = new CollectionExample();
	ex.execute();
\end{verbatim}

