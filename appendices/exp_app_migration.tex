\label{chap:exp_app_migration}
\section{Introduction}
To meet the clients needs in order to be able to migrate the system this manual will present the steps in the process to migrate the Genomizer system to a new server machine. This document will guide an experienced Linux user through the process of a migration from one server machine to a new one. The migration can be seen as a manual backup if the steps in \ref{sec:exp_steps} is followed. 

\section{Steps of migration}\label{sec:exp_steps}
\begin{enumerate}
	\item Run pg\_dump command on the server machine to migrate. This creating a database copy. For help, see \emph{Appendix \ref{sec:exp_app_clone}}
	\item Copy the \emph{/var/www/} folder on the current server machine and save the copy to a removable storage device.
	\item Install the new server machine with operating system and necessary software. For help with this see Appendix \ref{chap:exp_app_ubuntu} or \ref{chap:exp_app_debian}.
	\item When installation is done, paste the copied \emph{/var/www/} folder, in step 2, to the same location on the new server machine. 
	\item Insert the database copy into the new database on the new machine. For help, see \emph{Appendix \ref{sec:exp_app_inject}}
	\item Now the system is ready for the first startup. Make sure to have a new version of the Genomizer server.jar on the machine.
\end{enumerate}


