\thispagestyle{preface}

%A preface is best understood, I believe, as standing outside the book proper and being about the book. In a preface an
%author explains briefly why they wrote the book, or how they came to write it. They also often use the preface to
%establish their credibility, indicating their experience in the topic or their professional suitability to address
%such a topic. Sometimes they acknowledge those who inspired them or helped them (though these are often put into a
%separate Acknowledgments section). Using an old term from the study of rhetoric, a preface is in a sense an “apology”:
%an explanation or defense.
%
%The preface should contain information about:
%
%The purpose of the report
%The pre-history of the project
%Funding (if any)
%Major contents of the report
%Co-workers

This documentation describes the \appName\ project conducted during the spring of $2015$. The project is a part of the course in software engineering named \textit{Programvaruteknik(5DV151)} given by \textit{Umeå University}. The course is given to a mix of students. Some studying for a master and some for a bachelor degree in computer science. Hence different kind of knowledge exist within the student groups.

The documentation has two purposes. \textbf{1}) To give the possibility for tutors to assess the project and grade the students on their work. \textbf{2}) To describe the project and all parts of the developed system . 
The target audience of the system description are three subgroups: end users, system administrators and developers.

End users are epigenetic researchers. Developers are students of the course (current and future). System administrators are both students and personnel responsible for maintenance of the system.

The origin of the project is a perceived need from the epigenetic research department at \textit{Umeå University}. The wish is to make a more efficient pipeline for the computational parts of their research. An automated pipeline where knowledge of the parts involved is kept to a minimum. The automated pipeline would result in less time spent on  data entry and more time available for analysing data or conducting experiments.

\subsection*{Changes since version 2.3}\vspace*{-10pt}

\begin{tabularx}{\textwidth}{lX}
	Preface: & Introduced the preface chapter. \\
	Introduction: & Moved small parts to preface. Added some information. \\
	Service Description: & Restructured to be non-technical about each part of the system. \\
	Architectual design: & Moved as first chapter in development part as well as a bit more detailed. \\
	Meaning of the text targeted at:  & Decision taken in the design that needed to be right the first time to avoid time consuming alterations. \\
	Interaction Design: & Introduction in chapter is written that explains content of the chapter. \\
	Interaction Design: & Desktop section is more motivated on decisions, more pictures added. \\
	Interaction Design: & Web section has added pictures, motivations. \\
	System Design: & Desktop section fixed spelling, added pictures. \\
	Appendix: & Moved deployment and maintenance and added information about the virtual machines.\\
	Glossary:  & Glossary added after bibliography \\
\end{tabularx}

\subsection*{Acknowledgments}
\begin{itemize}
	\item Jonas Andersson: Great technical support and workflow support.
	\item Jonny Pettersson: Great support on the group dynamics.
	\item Jan-Erik Moström: Great feedback on documentation.
\end{itemize}


\subsection*{Developers spring 2015}\vspace*{-10pt}
\begin{tabularx}{\textwidth}{X  X  X}

	\begin{tabular}{l}
		\footnotesize\textbf{Data Storage}\\
		\footnotesize Nils Gustafsson \\
		\footnotesize Albin Råstander \\
		\footnotesize Jimmy Sihlberg \\
		\footnotesize Martin Larsson \\
		\footnotesize Erik Samuelsson \\
		\footnotesize Fredrik Uddgren \\ 
	\end{tabular} &
	%Processing
	\begin{tabular}{l} 
		\footnotesize \textbf{Processing}\\
		\footnotesize Adam Dahlgren Lindström \\
		\footnotesize Carl-Evert Kangas \\
		\footnotesize Emil Nylind \\
		\footnotesize Mikhail Glushenkov \\
		\footnotesize Saimon Marouki \\
	\end{tabular} & 
	% Business Logic
	\begin{tabular}{l} 
		\footnotesize\textbf{Business Logic}\\
		\footnotesize Johannes Ekman\\
		\footnotesize Alexander Frisk \\
		\footnotesize Tim Hedberg \\
		\footnotesize Mikael Johansson \\
		\footnotesize Mikael Karlsson \\
		\footnotesize Stefan Lindström \\ 
		\footnotesize Robin Ramquist \\
	\end{tabular} \\
	& &  \\
	% Desktop
	\begin{tabular}{l} 
		\footnotesize \textbf{Desktop}\\
		\footnotesize Viktor Bengtsson \\
		\footnotesize Maximilian Bågling \\
		\footnotesize Christoper Fladevad \\
		\footnotesize Jonas Hedin \\
		\footnotesize Petter Johansson \\
		\footnotesize Marcus Lööw \\
		\footnotesize Oscar Ottander \\
	\end{tabular} &
	% Mobile Applications 
	\begin{tabular}{l} 
		\footnotesize\textbf{Mobile Applications} \\
		\footnotesize Erik Berggren \\
		\footnotesize Jesper Bilander \\
		\footnotesize Victor Bylin \\
		\footnotesize Pål Forsberg \\
		\footnotesize Petter Nilsson \\
		\footnotesize Mattias Scherer \\		
	\end{tabular} & 
	%Web
	\begin{tabular}{l} 
		\footnotesize \textbf{Website}\\
		\footnotesize Ludwig Andersson \\
		\footnotesize Niklas Fires \\
		\footnotesize Andreas Günzel \\
		\footnotesize Pascal Hansson \\
		\footnotesize Patrik Hörnqvist \\
		\footnotesize Anna Jonsson  \\
		\footnotesize Björn Pers \\
	\end{tabular} 
\end{tabularx}

\footnotesize\textbf{Editorial Staff:} Mikael Johansson, Erik Samuelsson, Marcus Lööw, Jesper Bilander, Patrik Hörnqvist.

\today Mikael Johansson

