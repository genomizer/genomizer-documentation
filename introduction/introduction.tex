\appName\ is a system for storing and analyzing \term{DNA}-sequence data. It was designed for researchers in the field of epigenetics, who are interested in where on a \term{DNA} string certain proteins binds. In order to get this information, experiments are conducted and \term{raw} data files collected. These data files are then converted, in a series of steps, to files suitable for analysis. These files are hence refered to as \term{profile} data. \appName\ allows the researchers to upload \term{raw} files to a server and automate the generation of analysis data aswell as store the generated analysis data in a database for later access. 

The documentation contains three main parts. Introduction chapters that explain the goal of the project as well as a non-technical description of the project implementation. The development part of the document where the current implementation of each part of the project is explained how they look and work as well as an attempt to explain why certain design choices where done. Then finally there is a big collection of appendicies that goes deeper in their explanation of certain details of the implementation aswell as maintenance guides.

\nomenclature{raw}{Collection word for files that are the result from a \textit{DNA}-sequencing machine.}
\nomenclature{profile}{Data converted to a human readable file for analysis.}