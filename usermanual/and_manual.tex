In this section you will find instructions for the usage of the Genomizer android application. \ref{sec:and_start} describes how to start the application and \ref{sec:and_search} gives instructions on how to search for experiments.
\subsection{Start the application}\label{sec:and_start}
Localize the Genomizer icon in the android view of all your installed apps or
use a shortcut on one of your \term{desktop views}. Click the icon in order to start
Genomizer, you will be presented with the login screen.

To start working with your Genomizer app you have to login. If you have a
account with the service: insert your user name and password in the correspond-
ing boxes and press \click{Sign in} to access the application. If you are not yet
registered with the service, ask the system administrator for help with
the creation of your account.

\subsection{Selected files}
The selected files view is the first page the user is presented with and is the center of the application, used to store files during a session. (see \refer{fig:and_selected}) The page is divided into four different tabs that can be swiped (drag the finger from side to side).

\begin{figure}
	\addScaledImage{0.5}{and_selectedfiles.JPG}
	\caption{The selected files page}
	\label{fig:and_selected}
\end{figure}

\subsection{Search for files}\label{sec:and_search}
The search view offers a way to find data files with certain attributes. This view is accessed directly after logging in and can be reached at any time by pressing the
search icon on the action bar.

The annotation text fields can be filled in to find files matching the search
criteria. By checking the check box to the right of the annotation field, it gets
activated and appended as part of the search criteria. The functionality can
be used to conduct many similar searches by adding and subtracting criteria as
requested. The search is initiated by pressing \click{Search} at the bottom of the
page.

After pressing Search you will be  redirected to the search results view  that displays a list of available experiments that matches the search annotations. Every experiment is listed showing the experiment name. To receive more information about data files that are available for each experiment, click on an experiment in the list. By clicking an entry you will be taken to a new view displaying all available data files for that experiment. The data files are organised by: raw data, profile data and region data. Every data file got a checkbox next to it.  The checkboxes can be used to select files for conversion. When all files are selected click the \click{Send to conversion} button.
