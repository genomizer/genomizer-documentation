\subsection{How to run the app in Xcode}
In order to use the program, import the project from github into Xcode from the following repository:
\url{https://github.com/genomizer/genomizer-iOS.git} 

To compile and run the program, press \click{cmd+R}. A simulator will start and the login screen will be shown as seen in \refer{fig:ios_login}a.

\subsection{How to login}

\begin{enumerate}
\item Tap the \click{Gear} in the upper right corner and enter the url and port for the server you want to use and press \click{Done}. See \refer{fig:ios_login}a,c.
\item Tap on the \click{Username} textfield and enter your username.
\item Tap on the \click{Password} textfield and enter your password.
\item Tap on \click{Sign in} to sign in.
\end{enumerate}
A user gets logged in when accepted credentials are entered in the ‘username’ and ‘password’ fields and the ‘Sign in’ button is pressed. If incorrect credentials are entered, a popup message is shown, informing the user that the username or password is incorrect.

\begin{figure}[ht]
\addThreeImages{ios_login1.PNG}{a}{ios_login2.PNG}{b}{ios_login3.PNG}{c}
\caption{The login screen.}
\label{fig:ios_login}
\end{figure}
\FloatBarrier

\subsection{How to logout}
\begin{enumerate}
\item Tap \click{Gear}-symbol on the tab bar at the bottom of the screen and a \emph{Setting}-screen will appear. See \refer{fig:ios_more}
\item Tap \click{Logout} to logout.
\end{enumerate}

\begin{figure}[htb]
\addScaledImage{0.17}{ios_settings.PNG}
\caption{The settings screen.}
\label{fig:ios_more}
\end{figure}
\FloatBarrier

\subsection{How to search for experiments}

\begin{enumerate}
\item Tap on leftmost button(magnifying glass) on the tab bar which you can find on the bottom of the screen to get to the \emph{Search}-view.
\item Tap on the annotation you want to search for and a spinning wheel with options will appear from the bottom of the screen. See \refer{fig:ios_search}a.
\item Drag the wheel up or down to select the option you want or enter the value with the keyboard depending on the type of annotations is tapped.
\item Enable the annotation to use when searching by toggling the switch to the right of the annotation. See \refer{fig:ios_search}b.
\item Do steps (2)-(5) for more search criteria.
\item Tap \click{Search} to search.
\end{enumerate}

\begin{figure}[ht]
\addThreeImages{ios_search1.PNG}{a}{ios_search2.PNG}{b}{ios_search4.PNG}{c}
\caption{The search screen.}
\label{fig:ios_search}
\end{figure}
\FloatBarrier

\subsection{How to use advanced search}

\begin{enumerate}
\item Tap on leftmost button(magnifying glass) on the tab bar which you can find on the bottom of the screen to get to the \emph{Search}-view.
\item Tap on the symbol to top right of the screen and a new view will appear with the title \emph{Advanced Search}. See \refer{fig:ios_search}c.
\item Write your search criteria in PubMed-style and tap \click{Search}
\end{enumerate}
The annotations you select on the search-screen will also show in advanced search.

\subsection{How to process files}


\begin{enumerate}
\item Search for experiments
\item In \emph{Search Results}-screen tap on an experiment and the \emph{Files}-screen will appear showing the files which belongs to the experiment. See \refer{fig:ios_files_view}a
\item Tap the \click{Plus}-symbol next to the file you want to process.
\item Do step (3) for every file you want to process. Same file can be used more than once. A counter will appear next to the Plus-symbol to keep track on how many times the file will be used in the process-step, see \refer{fig:ios_files_view}b.
\item Tap the \click{Process}-button and \emph{Make a process}-view will appear with every file you have selected. See \refer{fig:ios_make_process_view}a
\item Tap \click{Add Process} and select what kind of process you want to do on the selected files.
\item After you have selected a process the input files and output files will appear with the selected process separating them. You can add parameter values by tapping the parameter fields below the input files. The output files can be renamed by tapping on the filename. See \refer{fig:ios_make_process_view}b
\item Do step (6) to create a sequence of processes to be made on the selected files. See \refer{fig:ios_make_process_view}c for an example of a process sequence.

\item Tap \click{Done}-button to send the sequence of processes to the server.
\end{enumerate}


\begin{figure}[htb]
\addTwoImages{ios_process_files.jpg}{a}{ios_files_added.jpg}{b}
\caption{Files view.}
\label{fig:ios_files_view}
\end{figure}
\FloatBarrier

\begin{figure}[htb]
\addThreeImages{ios_empty_make_process.jpg}{a}{ios_one_process.jpg}{b}{ios_many_process.jpg}{c}
\caption{Create processes.}
\label{fig:ios_make_process_view}
\end{figure}
\FloatBarrier


\subsection{How to set which annotation to be visible on Search Results}

\begin{enumerate}
\item In Search Results view, tap \click{Edit} and \emph{Select Annotations}-screen will appear
\item Select which annotations to show by toggle the switch next to each annotation.
\item Tap \click{Back} to go back to \emph{Search Results}. See \refer{fig:ios_searchResult}a-c
\end{enumerate}

\subsection{How to change the order which search results appear in}

\begin{enumerate}
\item In Search Results view, tap \click{Edit} and \emph{Select Annotations}-screen will appear
\item Under the \emph{Sort by} header, drag-and-drop the annotation-names in the order you wish to sort the search result
\end{enumerate}

\begin{figure}[ht]
\addThreeImages{ios_result1.PNG}{a}{ios_result2.PNG}{b}{ios_result3.PNG}{c}
\caption{Select annotation}
\label{fig:ios_searchResult}
\end{figure}
\FloatBarrier

\subsection{How to view process status on the server}

\begin{enumerate}
\item Tap \click{Process}-symbol (the percentage symbol) to view the processes on the server.
\item To refresh the view, drag the view down until an activity indicator icon is visible below the title of the screen and release. See \refer{fig:ios_processes}
\end{enumerate}

\begin{figure}[htb]
\addScaledImage{0.3}{ios_processes1.PNG}
\caption{The process screen.}
\label{fig:ios_processes}
\end{figure}
\FloatBarrier

%\subsection{How to view information about a file}

%\begin{enumerate}
%\item Tap \click{information}-symbol next to the file to view information of the file.
%\item Close by tapping \click{Close}. See \refer{fig:ios_files1}a-b
%\end{enumerate}

%\begin{figure}[htb]
%\addTwoImages{ios_files4.PNG}{a}{ios_files2.PNG}{b}
%\caption{The files screen.}
%\label{fig:ios_files1}
%\end{figure}
%\FloatBarrier


