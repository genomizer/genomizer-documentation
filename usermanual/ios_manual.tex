\subsection{How to run the app in Xcode}
In order to use the program, import the project from github into Xcode from the following repository:
\url{https://github.com/genomizer/genomizer-iOS.git} 

To compile and run the program, press \click{cmd+R}. A simulator will start and the login screen will be shown as seen in \refer{fig:ios_login}a.

\subsection{How to login}

\begin{enumerate}
\item Tap the settings button in the upper right corner and enter the url and port for the server you want to use and press \click{Done}. See \refer{fig:ios_login}a,c.
\item Tap on the \click{Username} textfield and enter your username.
\item Tap on the \click{Password} textfield and enter your password.
\item Tap on \click{Sign in} to sign in.
\end{enumerate}
A user gets logged in when accepted credentials are entered in the ‘username’ and ‘password’ fields and the ‘Sign in’ button is pressed. If incorrect credentials are entered, a popup message is shown, informing the user that the username or password is incorrect.

\begin{figure}[ht]
\addThreeImages{ios_login1.PNG}{a)}{ios_login2.PNG}{b)}{ios_login3.PNG}{c)}
\caption{The login screen.}
\label{fig:ios_login}
\end{figure}
\FloatBarrier

\subsection{How to logout}
\begin{enumerate}
\item Tap \click{Gear}-symbol on the tab bar and a \emph{Setting}-screen will appear. See \refer{fig:ios_more}
\item Tap \click{Logout} to logout.
\end{enumerate}

\begin{figure}[htb]
\addScaledImage{0.17}{ios_settings.PNG}
\caption{The settings screen.}
\label{fig:ios_more}
\end{figure}
\FloatBarrier

\subsection{How to search an experiment}

\begin{enumerate}
\item Ensure that you on the search-screen by looking at the title on the top. If it says \emph{Search} you can skip (2). 
\item Tap on leftmost button(magnifying glass) on the tab bar which you can find on the bottom of the screen.
\item Tap on the annotation you want to search for and a spinning wheel with options will appear from the bottom of the screen. See \refer{fig:ios_search}a.
\item Drag the wheel up or down to select the option you want.
\item Enable the annotation to use when searching by toggling the switch to the right of the annotation. See \refer{fig:ios_search}b.
\item Do 2-4 for more search criteria.
\item Tap \click{Search} to search.
\end{enumerate}

\begin{figure}[ht]
\addThreeImages{ios_search1.PNG}{a)}{ios_search2.PNG}{b)}{ios_search4.PNG}{c)}
\caption{The search screen.}
\label{fig:ios_search}
\end{figure}
\FloatBarrier

\subsection{How to use advanced search}

\begin{enumerate}
\item Ensure that you on the search-screen by looking at the title on the top. If it says \emph{Search} you can skip (2). 
\item Tap on leftmost button(magnifying glass) on the tab bar which you can find on the bottom of the screen.
\item Tap on the symbol to top right of the screen and a new view will appear with the title \emph{Advanced Search}. See \refer{fig:ios_search}c.
\item Write your search criteria in PubMed-style and tap \click{search}
\end{enumerate}
The annotations you select on the search-screen will also show in advanced search.

\subsection{How to convert raw files to profile}


\begin{enumerate}
\item Search for experiments
\item In \emph{Search Results}-screen tap on an experiment and the \emph{Files}-screen will appear showing the files which belongs to the experiment.
\item Tap the \click{Star}-symbol to the right of the file you want to process, to add them to your workspace. Selected files are shown by having a yellow star on them. See \refer{fig:ios_files3}
\item Tap on the \click{Star}-symbol(the workspace) on the tab bar to view all selected files. See \refer{fig:io_selectedFiles}a
\item Toggle the switch on the files you want to process.
\item Tap \click{Select task to perform} and then tap \click{Convert to profile} and the \emph{Convert}-screen will appear. See \refer{fig:ios_selectedFiles}b and \refer{fig:ios_convert}a
\item Enter the settings you want to use when converting.
\item Tap \click{Convert} to convert. See \refer{fig:ios_convert}b-c
\end{enumerate}

\begin{figure}[htb]
\addScaledImage{0.17}{ios_files3.PNG}
\caption{The files screen.}
\label{fig:ios_files3}
\end{figure}
\FloatBarrier

\begin{figure}[htb]
\addTwoImages{ios_selected1.PNG}{ios_selected2.PNG}
\caption{Workspace.}
\label{fig:ios_selectedFiles}
\end{figure}
\FloatBarrier

\begin{figure}[htb]
\addThreeImages{ios_convert1.PNG}{a)}{ios_convert2.PNG}{b)}{ios_convert3.PNG}{c)}
\caption{The select task screen.}
\label{fig:ios_convert}
\end{figure}
\FloatBarrier


\subsection{How to set which annotation to be visible on Search Results}

\begin{enumerate}
\item In Search Results view, tap \click{Edit} and \emph{Select Annotations}-screen will appear
\item Select which annotations to show by toggle the switch next to each annotation.
\item Tap \click{Back} to go back to \emph{Search Results}. See \refer{fig:ios_searchResult}a-c
\end{enumerate}

\begin{figure}[ht]
\addThreeImages{ios_result1.PNG}{a)}{ios_result2.PNG}{b)}{ios_result3.PNG}{c)}
\caption{Select annotation}
\label{fig:ios_searchResult}
\end{figure}
\FloatBarrier

\subsection{How to remove files from workspace}
\begin{enumerate}
\item Tap \click{Star}-symbol on the tab bar to view the workspace.
\item Toggle the switch next to the file you want to remove.
\item Tap \click{Trash can}-symbol to the upper right corner to remove the selected files. See \refer{fig:ios_selectedFiles}
\end{enumerate}

\subsection{How to view process status on the server}

\begin{enumerate}
\item Tap \click{Process}-symbol(the symbol between the star and the trash can) to view the processes on the server.
\item To refresh the view, drag the view down until an activity indicator icon is visible below the title of the screen and release. See \refer{fig:ios_processes}
\end{enumerate}

\begin{figure}[htb]
\addScaledImage{0.17}{ios_processes1.PNG}
\caption{The select task screen.}
\label{fig:ios_processes}
\end{figure}
\FloatBarrier

\subsection{How to view information about a file}

\begin{enumerate}
\item In either workspace tab or files view, tap \click{information}-symbol next to the file to view information of the file.
\item Close by tapping \click{Close}. See \refer{fig:ios_files1}a-b
\end{enumerate}

\begin{figure}[htb]
\addTwoImages{ios_files4.PNG}{ios_files2.PNG}
\caption{The files screen.}
\label{fig:ios_files1}
\end{figure}
\FloatBarrier

















