This chapter will present an overview of the services that the \appName\ system will provide. 

\section{Usage}
The service will a central database which can be accessed from multiple devices (Windows, Linux, OSX, Android and iPhone) using different clients. Using the desktop clients, users can upload genome data and annotate it with info about the data.   

Advanced searches or different annotation can be performed in the database to find data which has previously been uploaded to the database. When some interesting data is found in the database it can be downloaded in a variety of file formats. If a file does not exist in a specific file format it will be converted before it's sent to the user.

\section{Genome release}
The researchers will be able to upload chain files between different genome releases and genome references for different species. When a chain file is uploaded it will be possible to convert profile and region data between different releases.

\section{Analyze}
The user will be able to conduct different types of analysis on different datasets, the actual analysis will be calculated on the server to avoid heavy workload on the clients.

\section{Visualization}
The system will have support for opening a session in Integrated Genome browser directly from the interface.

\section{Workspace}
To gather, organize and share data the concept workspace has been introduced. A user can save raw data, profile data and region data as well as analysis results in a workspace. The workspace is saved on  the server so that the user can continue their work on another place as long as internet is available.  Another feature with workspaces is that it can be shared between users, this is valuable when users work on different locations.

\section{Mobile}
In the mobile application the users will be able to start downloading new datasets from other databases to the genomizer database, start or schedule processing of data and view the visualization of analysed data.