


This chapter will present an overview of the services that the \appName\ system will provide. 

\section{Usage}



\begin{figure}[h]
\addImage{genomizerDiagramServiceDescription.png}
\caption{Communication diagram of the product}
\label{con_serviceDescription}
\end{figure}
	
In order to give the users flexibility when using the service there will be clients for many different platforms (Windows, Linux, OSX, Web, Android and iPhone) that they can login on (See \refer{con_serviceDescription}). 
When a user then chose a given task, for example start raw-to-profile processing that task is sent by internet to the server as shown in \refer{con_serviceDescription} which will handle the request and send a response back to the user.

The main purpose of the Genomizer system is to centralize all data. For this purpose a user with a desktop client can upload data with flexible annotation to the data storage on the server. 
Advanced searches on different annotation can be performed in the database to find data which has previously been uploaded to the database. When some interesting data is found in the database it can be downloaded to the users local computer. 
After searching the user can also decide to select files and send requests to the server to process said files.


\section{Genome release}
The researchers will be able to upload chain files between different genome releases and genome references for different species. When a chain file is uploaded it will be possible to convert profile and region data between different releases. 

\section{Analyze}
The user will be able to conduct different types of analysis on different datasets, the actual analysis will be calculated on the server to avoid heavy workload on the clients. The primary analysis is between raw data and profile data through many steps including smoothing. Because a user sometimes might not want to do all the steps involved when processing raw data into profile data, the user can choose how far the processing should go.

\section{Mobile}
The user can search for files in the database in the mobile application. 
After they have found the files they want they can schedule them for conversion or processing.

In the mobile application the users will be able to start downloading new datasets from other databases to the genomizer database, start or schedule processing of data and view the visualization of analysed data.

\section{Annotations}
A user with access to the systemadmin page can add or delete annotations in the database.


