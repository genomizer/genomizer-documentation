


This chapter will present an overview of the services that the \appName\ system provides. 

\section{Usage}

\begin{figure}[h]
\addImage{genomizerDiagramServiceDescription.png}
\caption{Communication diagram of the product}
\label{con_serviceDescription}
\end{figure}
	
In order to give the users flexibility when using the service there are clients for many different platforms (Windows, Linux, OSX, Web, Android and iPhone). 
When a user chooses a given task, for example start \term{raw} to \term{profile} processing, that task is sent by internet to the server as shown in \refer{con_serviceDescription} which will handle the request and send a response back to the user.

The main purpose of the \appName\ system is to centralise all data. To enable this a user can annotate and upload data to the server using both desktop and web based clients. 
Advanced database searches can be performed on the annotations to find previously uploaded data. When the required data is found the user can choose to download the files or schedule them for processing on the server.

\section{Processing}
Users can schedule various processing procedures on different data sets. This is carried out on the server to avoid heavy workload on the clients. The processing carried out between \term{raw} data and \term{profile} data involves a number of different steps. The user can choose which steps are carried out and the various parameters used.

\section{Genome releases}
Users can upload new genome release files and use them in the processing of \term{raw} to \term{profile} data.

\section{Mobile}
The mobile applications enable the searching of files in the database and the scheduling of processing procedures for the conversion of \term{raw} to \term{profile} data.

\section{Annotations}
A user with access to the \term{Systemadmin} page can add or delete annotations used to store information about experiments.


