This section explains implementation details of certain bits of the communication/control part of the system.
\subsubsection{Doorman}
The doorman is a class which greets all incoming connections and requests. The doorman creates a RequestHandler and a server context so that the requests can be correctly handled.\\
\\
\subsubsection{RequestHandler}
The RequestHandler class checks the server context to determine what Command is to be created and what action to take. If for instance the HTTP header is \textbf{GET /experiment} a \term{GetExperimentCommand} is created. The command is afterwards validated and executed. Uploading and downloading is handled differently. Instead a downloadHandler/uploadHandler is used to handle the exchange.\\
\\
\subsubsection{Authorization}
The communication between a client and the server is authorized by a user-unique token which is created when the user sends a login request. A token is created when a user has logged in successfully and the token is sent back to the user so that the user can thereafter use this in future requests. The token created when a user sends a login is stored in the server memory until the user sends a logout request.
\begin{figure}[h]
\addImage{com_authorization.png}
\caption{1. The user sends a login request without any authorization token. 2. The server checks the given password. 3. The server creates a unique token for the user. 4. Server sends the token back to the user in a response. 5. Now that the user has a unique token, the token is placed in the header whenever the user sends another request.}
\label{fig:com_authorization}
\end{figure}
\subsubsection{Removing inactive tokens}
The server has a function which removes inactive tokens after a set limit of time. This is done because a client sometimes skips sending a logout request when shutting down the client program. The InactiveUuidsRemover class is used to achieve this goal. In a thread it sleeps for one hour before checking all clients. If any client hasn't sent a request for 24 hours, the client token is removed from the server memory. 
\\
This feature may be turned of with the flag "-nri".
\subsubsection{Command object}
The command object represent a specific command. It is created from the RESTful header and/or the \json/ body sent from the client. The \json/ API provides methods for automatic parsing of the \json/ body into an object. The fields in the command object created must match the attributes in the \json/ body. This match is case sensitive
\paragraph{Execute}
Every command object must implement a execute method. This method is the part of the command which uses the system interface to perform the task that corresponds to the command.\\
\\
The execute method returns a response object which is sent up to the RequestHandler which then sends the response to the client. 
\paragraph{Validation}
Every command must implement a validate method. This method is run after the command is created but before the command is executed.\\
\\
The validate method throws ValidateExceptions if any information given by the client is in an incorrect format. The validation is used to prevent unnecessary communication.
\subsubsection{Processing handling}
For processing of data for instance multiple instances of a raw to profile convertion a queue is used. A client sends a list of processes to run by the command PutProcessCommands. The list of processes is placed in a ProcessPool which can start and cancel processing. All lists of processes that are placed in the queue are executed one at a time in the order first in first out. If a certain list of processes seems to take too long and is not prioritized the list can be cancelled through the CancelProcessCommand. Note that one single process in the list can not be individually cancelled but the entire list inputed through that particular PutProcessCommands can only be cancelled.\\
\\ 
\subsubsection{Response object}
There are different response objects for different kind of responses since the form of the response to the client depends on the command the client initially sent.\\
\\
The response object contains all the data necessary to create a RESTful header and a \json/ 
body for the response.
\subsubsection{Testing}
Testing has been done in multiple steps. The first step is unit testing, where individual methods are tested. This is often difficult due to the fact that the responsibility of handling client requests is shared by multiple classes. To catch these test cases a client dummy has been frequently used, which is the next step. 
It simulates a client by sending HTTP requests and examines the response from the server. It is used
manually to test a particular use case, and to see that the server behaves as intended for that request.
After a feature has passed the client dummy it is pushed to a test server, where it is open for other clients to test and debug.
If no bugs are found the feature is declared complete and can be released.
\subsubsection{Travis}
To make sure of continous integration Travis is used with the GitHub project. Whenever a change is pushed to a branch of the project Travis will try to build the project and run the JUnit tests. No branches are allowed to be merged with others unless the Travis tests succeeded. 