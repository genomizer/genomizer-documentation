The following text describes the different classes the server uses to communicate with the database and update the file system. All the communication with the database happens through the DatabaseAccessor-class, which in turn uses several helper classes that contains the actual logic. 

\begin{figure}[h]
\addImage{dat_overview_schema.png}
\caption{AMEN DURÅA!}
\label{fig:dat_overview_schema}
\end{figure}

\subsubsection{Annotation}
When a user wants the properties of an \term{annotation} they will retrieve this object. It holds information about the data type, label, forced annotation, default value and annotation choices for drop down menus. With this object the graphical interfaces can set up a search view dynamically.

\subsubsection{Experiment}
A container of an experiments annotations and their values. The annotation labels and values are contained in a \term{hashmap} with label as key. All files corresponding to the actual experiment lies in a list of \term{FileTuple} objects.

\subsubsection{FileTuple}
A simple class that holds all information of a file in variables found in the database. Holds links for download and upload of the specific file.

\subsubsection{FilePathGenerator}
The creation of folders is handled by the \texttt{FilePathGenerator} class. When a new experiment is added to the database a folder is also created for the new experiment. Sub folders are also created in preperation for the uploading of files.

When a user requests to upload a file, the \term{FilePathGenerator} class will, if required, generate a new folder to house the file. 

When a new \term{genome release} is added to the database, the \term{FilePathGenerator} will create a folder to house the associated files. Genome releases are divided into folders corresponding to species.

A new folder is also created for each process request and houses the resulting file set.

\subsubsection{PubMedToSQLConverter}
This class converts a \term{PubMed} string to a \term{SQL query}. A typical \term{PubMed} search string takes the form: \texttt{raw[FileType] AND Per[Author]}. In this case the search is for \term{all raw files that Per created}. The user enters the annotation labels and values together with the logical operators \term{AND}, \term{OR} and \term{NOT}. Parentheses are used for disambiguation. 

All the variables in the PubMed string are bound to variables in the \texttt{WHERE} section of the SQL query to avoid SQL injection. 

When the \texttt{search} method is called in the \texttt{DatabaseAccessor}, the \term{PubMed} string is checked for file attributes by calling the \texttt{hasFileAttributes} method in the \texttt{PubMedToSQLConverter}. This is done so that even empty experiments are returned when searches do not contain a file specific attribute.

\subsubsection{DatabaseAccessor}
A class that serves as an API for the Genomizer database - it handles all connections and queries to the database. The methods simplify queries to the database by removing the need to write SQL in any other packages. The \texttt{DatabaseAccessor} utilizes its subclasses as shown in \ref{fig:dat_dbac} where the methods are divided up into more specific areas. \term{Prepared Statements} are created from queries and parameters to avoid SQL injection. Methods that have some resemblance are grouped together for easier navigation.

\subsubsection{Testing}
WARNING! Do not run any of the tests found in the \texttt{database} package on a database that is in use. All tuples are removed from the database upon test completion. Every instance of unit testing should start with an empty database and finish with an empty database to avoid dependency between tests.

The database package was created using \term{Test Driven Development} (TDD). The full test suite can be found in the \texttt{database.test} package.

All the unit tests utilize the \texttt{TestInitializer} class. This simplifies the process of connecting to the test database, filling it with test tuples and clearing the test database and closing the connection when the test class is finished. 
The individual unit tests can be found in the \texttt{database.test.unittests} package. The scripts for adding the test tuples and clearing the test database tables can also be found in the \texttt{sql} package.