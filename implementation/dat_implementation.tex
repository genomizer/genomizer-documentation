The following text describes the different classes the Genomizer server uses to communicate with the database and the file system. All the communication with the database happens through the DatabaseAccessor-class, which uses several helper classes that contain methods with the actual logic. These relationships are visualized in \refer{fig:dat_overview_schema}. 

\begin{figure}[h]
\addImage{dat_overview_schema.png}
\caption{Conceptual overview - The implementation of data access and storage}
\label{fig:dat_overview_schema}
\end{figure}

\subsubsection{DatabaseAccessor}
DatabaseAccessor is a class that serves as an API for the Genomizer database - it handles all connections and queries to the database. The methods simplify queries to the database by removing the need to write SQL in any other packages. For more details see the class diagram, \refer{fig:dat_dbac}, in \refer{chap:dat_umls}.

\subsubsection{Containers}
The classes grouped under the name Containers are used for representation of domain specific objects. An \term{Experiment} is a container for an experiment's annotations and their values as well as a list of files corresponding to that experiment. These files are contained in \term{FileTuple} objects which holds information regarding filename, type, size etc. The \term{Annotation} class contains information about label, whether or not the label is required, the default value and annotation choiches for drop down menus. More information can be found in the class diagram \refer{fig:dat_containers} in \refer{chap:dat_umls}.

\subsubsection{Methods}
The \term{DatabaseAccessor} uses helper classes to execute SQL queries. These helper classes are grouped by their area of responsibility:
\begin{itemize}
\item Annotation methods
\item Experiment methods
\item File methods
\item Genome methods
\item User methods
\end{itemize}
These classes execute the basic SQL functions Create, Read, Update and Delete (CRUD). The use of \term{Prepared Statements} (also known as parameterized queries) safeguards against SQL injection.

\subsubsection{Helper classes}
The creation of folders is handled by the \texttt{FilePathGenerator} class. Folders are created for a new experiment when the experiment is added to the database, including subfolders in preparation of files to be uploaded.

When a user requests to upload a file the \term{FilePathGenerator} class will, if required, generate a new folder to house the file.

When a new \term{genome release} is added to the database the \term{FilePathGenerator} will create a folder to house the associated files. Genome releases are divided into folders corresponding to species.

A new folder is also created for each process request and houses the resulting file set.

\subsubsection{PubMedToSQLConverter}
This class converts a \term{PubMed} string to a \term{SQL query}. A typical \term{PubMed} search string takes the form: \texttt{raw[FileType] AND Per[Author]}. In this case the search is for \term{all raw files that Per created}. The user enters the annotation labels and values together with the logical operators \term{AND}, \term{OR} and \term{NOT}. Parentheses are used for disambiguation. 

All the variables in the PubMed string are bound to variables in the \texttt{WHERE} section of the SQL query to avoid SQL injection. 

When the \texttt{search} method is called in the \texttt{DatabaseAccessor}, the \term{PubMed} string is checked for file attributes by calling the \texttt{hasFileAttributes} method in the \texttt{PubMedToSQLConverter}. This is done so that even empty experiments are returned when searches do not contain a file specific attribute.



\subsubsection{Testing}
WARNING! Do not run any of the tests found in the \texttt{database} package on a database that is in use. All tuples are removed from the database upon test completion. Every instance of unit testing should start with an empty database and finish with an empty database to avoid dependency between tests.

The database package was created using \term{Test Driven Development} (TDD). The full test suite can be found in the \texttt{database.test} package.

All the unit tests utilize the \texttt{TestInitializer} class. This simplifies the process of connecting to the test database, filling it with test tuples and clearing the test database and closing the connection when the test class is finished. 
The individual unit tests can be found in the \texttt{database.test.unittests} package. The scripts for adding the test tuples and clearing the test database tables can also be found in the \texttt{sql} package.