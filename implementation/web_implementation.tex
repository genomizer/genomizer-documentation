

\subsection{Frameworks}
\label{sec:web_frame}
To ease implementation, a couple of frameworks have been used. The frameworks are described briefly below.
\subsubsection{Backbone}
Backbone\cite{web_1} is a light-weight framework that loosely follows the MVC pattern. Out of the MVC components, Backbone only has models and views, and the view behaves much like a combination of both a view and a controller. \class{Models} are the parts of code that retrieve and populate data. \class{Views} are the HTML representation of models, and they change as models change.

\begin{example}
When the \class{Experiment} model is populated, which may happen when the model fetches data from the server, it is immediately presented on the view that contains that experiment. The view itself does not have to manually get any data from the \class{Experiment} model.
\end{example}

Backbone makes use of \class{Events}, where other objects can trigger events and listen to them, which is an effective way to promote decoupling between components. It also uses \class{Collections}, that are ordered sets of models. A collection will automatically be provided with underscore array and collection methods for convenient set manipulations (you can, for example, loop through a collection with \class{.each()} instead of writing a for-loop). Backbone is used because it allows more structure in the web application. With more structure, it is easier to collaborate as the work can be divided - keeping the Javascript code in various model, collection and view files.

\subsubsection{Bootstrap}
Bootstrap\cite{web_2} is a front-end framework that contains HTML and CSS-based design templates for typography, buttons, forms, navigations, and the like. Instead of creating buttons from scratch, deciding on colors, how big they are, and micromanaging how they fit with everything else on the page, bootstraps templates that handles all of that, leaving the developers able to focus on architecture. Bootstrap is used to save time on development and make the design of the web app easily customizable.

\subsubsection{RequireJS}
RequireJS\cite{web_5} is a file and module loader for Javascript. RequireJS lets files require other files much like \texttt{\#include} in Java. This is very handy for the programmer. It is used because it helps to structure the application.

\subsubsection{JQuery}
JQuery\cite{web_6} is a popular code library that handles AJAX calls and DOM manipulation, and makes the code more compact and readable.


\subsection{Technologies used}
A couple of technologies have been used in the development and are described below.

\subsubsection{AJAX}
AJAX\cite{web_3} (Asynchronous Javascript and XML) is a technique for creating fast and dynamic web pages. Despite the name, the use of XML is not required; \json\ is often used instead, which is the case in the Genomizer web app. AJAX allows web pages to be updated asynchronously by exchanging small amounts of data with the server, so that you only update parts of a webpage without having to reload the entire page (like websites that do not use AJAX have to). For example, when the search button is clicked in the navigation bar of the web app, only the bottom half of the website is updated, and displaying the search view. The navigation bar does not have to be reloaded, but remains as it is on top.

\subsubsection{\json}
\json\cite{web_4} (Javascript Object Notation) is a format that is primarily used to transmit data between a server and a web application instead of using XML or other formats.
\json\ is formatted as easily readable text consisting of attribute-value pairs.
\json\ was used in this application because \json\ uses the same syntax as Javascript and therefore no parsing is needed, as opposed to the usage of e.g. XML. \json\ also works well together with Backbone as it has integrated methods using the \json\ format.

\subsection{Testing frameworks}
Three libraries are used to make testing easier: Chai, Mocha and Sinon. Together they let the developers make a page for testing where all tests and results will be shown visually.
These libraries or testing frameworks will be discussed below.
\subsubsection{Chai \& Mocha}
Mocha\cite{web_8} is a test framework while Chai\cite{web_7} is an expectation framework. While Mocha setups and describes test suites, Chai provides convenient helpers to perform all kinds of assertions against Javascript code. We use these frameworks to do unit testing on our models and collections.

\subsubsection{Sinon}
Sinon\cite{web_9} is a framework used to “fake environment”. When doing unit testing, we do not want to depend on things that are external to the unit of code that we are testing. Sinon can be used for stubbing and mocking external dependencies and to keep control on side effects against them. For example, Sinon can be used to create spies to see if an event has been triggered, and to create fake servers that respond with fake pre-planned responses to queries.

\subsection{Web app tests}
Unit tests have been performed on all model and collection files that contain non-trivial functions. All unit tests can be found in the root folder under \filePath{/tests/}, more specifically \filePath{/genomizer-web/tests/}. To run the tests, simply open the index.html in a web browser and they will run. The views have not been unit tested since that is overly complicated; instead they have been continuously manually tested throughout the development process. In addition to these simple development tests, more official system tests have also been done by the desktop group.
