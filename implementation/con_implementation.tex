
\subsection{SmoothingAndStep}

The basic algorithm is a dynamic arrayList which carries the rows that are relevant at a given time, smoothing on the first row is performed. The newly smoothed value is shifted out and replaced with a fresh row. This becomes a dynamic window that traverses the entire file one row at a time.

\subsubsection{Methods}

\begin{itemize} 
\item smoothing : The one public method of the class. It controls the whole process and calls the other methods. It takes in the following parameters:
\begin{itemize}
\item int[ ] params: An array with 5 integers representing parameters.
     params[0]: Window Size, the nu 
mber of signal values that the smoothing
      		  should be calculated on. \newline
      params[1]: Wether the smoothing should be trimmed mean (0) or median (1)\newline
      params[2]: Minimum numbers to smooth. A number that says how many signal
      		  values the program at least need in order to smooth one row.
      		  This number must be smaller than windowSize. \newline
      params[3]: Can either be 1 or 0. If 1 the program will calculate the
      		  total mean value for all rows and print those. \newline
      params[4]: Print zeroes. If the program should print rows where the
      		  signal value is 0 the flag should be (1), if (0) the program
      		  will not print the zeroes. \newline
\item String inPath: A filepath to the source file.
\item String outPath: A filepath to either an existing file to be overwritten or of a location and name that will become the path to a newly created file.
\item int stepSize:  An integer larger than 0 that tells if there should be stepping. No stepping will be done if the number is 1.
\end{itemize}

The method will also return the total mean of every row in the file if that flag is set properly.

\item smoothOneRow: Checks wether smoothing should be trimmed mean or median and calls the corresponding method, after this is done it calls the method that writes to the new file.

\item smoothTrimmedMean: Extracts the first position from the data array and initiates it's value to min and max values. We do this because trimmed mean means that we should remove the largest and smallest number from the mean value in order to get a more reliable/stable result. We then check that we have more numbers in the data array than the minimum numbers to smooth number. In order to avoid doing unnecessary calculations. 

\item smoothMedian: This method tries to fill an array with window size number of signal values and then pass this array to a method that finds which number is the median.

\item writeToFile: This method does three different things. It check wether we should print zeroes in the outfile. It also check wether the current position is divisible with stepSize to determine if the row should be written to the outfile or skipped.  After these two checks it either writes the row to the new file or not.

It also check wether we want to print the total mean of the whole file and/or if we should then it counts up the proper variables.

\item shiftLeft: Removes the first row from the data array and adds one row to the end of it.  It then checks wether the new row is of a different chromosome than the others, if so it calls the special method chromosomeChange.

\item chromosomeChange: This method knows that the last element in the data array is a new chromosome. It then reads and smooths as many rows as it can before hitting the cutoff number (minimum number of rows to smooth). It then writes and removes these values from the data array aswell. It's important to note that so far it doesn't add new values to the array. Afterwards the method tries to refill the array with the new chromosome until it has window size number of rows.



\end{itemize}



 