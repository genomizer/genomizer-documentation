This section focuses on the communication  between classes and how the Android application works.
\subsubsection{Login request}
	When the user starts the application and is prompted for a login, the the following sequence of actions is performed by the system. (See \refer{fig:and_loginseq})	\\
	
	\begin{itemize}
		\item
			User starts the application and is prompted with a login.
			Then the user-name and password is passed on to the LoginFragment.
		\item
			LoginFragment sends a login request to the ComHandler, with the user-name and password.
		\item
			The ComHandler initializes Communicator and sends a setupConnection command to verify that connection can be made. Then initializes the MsgFactory and request a login-package using the createLogin method.
		\item
			MsgFactory responds to ComHandler with a login-package in JSon format. Then the ComHandler calls the sendRequest method in Communicator with the login-package and waits for reply.
		\item
			The Communicator connects to the remote Genomizer server with a HTTP-request containing the login-package as a JSon object. If the search is valid the server will respond with code 200 and a user-token as a JSon-object.
		\item
			The JSon-object is then returned to the ComHandler which will store the token and return a boolean to the LoginFragment informing if the login was successful or not.
		\item 
			If the response was true the LoginFragment will startup the SearchFragment and present that view to the user. The Login Fragment will be terminated at that point. 
			
	\end{itemize}

	\begin{figure}[h]
		\addImageVertical{and_login_seq.jpg}
		\caption{Sequence diagram for a login-request}
		\label{fig:and_loginseq}
	\end{figure}
	
\subsubsection{Search request}
	When the user sends a search request using the search view in the application, the following sequence of actions is performed by the system.
	(see \refer{fig:and_searchseq})
	
	\begin{itemize}
		\item
			User will make a search request on the screen and press on the search button. That will trigger the SearchFragment to startup the ExperimentListFragment with the search string sent in as a Intent-Extra variable.
		\item
			The ExperimentListFragment will then initialize the ComHandler and call the search method with the search-list provided by the SearchFragment.
		\item
			ComHandler initializes the Communicator and determines if a connection can be made.
		\item
			If connection can be made the ComHandler initialize the MsgFactory and calls the createRegularPackage method, which will return a pre-formatted JSon-object to be used with the search request to the server.
		\item
			ComHandler calls Communicator using the sendRequest method passing on the JSon-object containing the search-list, and waits for the reply from Communicator.
		\item
			Communicator connects to the Genomizer server with a HTTP request containing a JSon object with the search-list. The server will respond with code 200 and a JSon-object with the search result from the server.
		\item
			Communicator will reconfigure the JSon object to a GenomizerHttpPackage object to preserve both the head and body from the server response. Then the GHttp-object is returned to the ComHandler.
		\item
			The ComHandler will initalize the MsgDeconstruct and send the collected JSon-object containing the search result from the server to the deconSearch method.
		\item
			MsgDeconstruct will parse the JSon-object to an ArrayList of experiments, and return that to the ComHandler.
		\item
			ComHandler will then return the search-results to the ExperimentListFragment that will present the results for the user on the screen.
	\end{itemize} 

	\begin{figure}
		\addImageVertical{and_search_seq.jpg}
		\caption{Sequence diagram for a search-request}
		\label{fig:and_searchseq}
	\end{figure}
\subsubsection{Testing}
Testing has been done successively and the foremost type of testing has been JUnit tests. Although regular running and logs have been done too.

All fragments mentioned in \ref{sec:and_classdescription} have been tested visually and through running as no viable way of unit testing them was found.

Most of the classes labeled model in \ref{sec:and_classdescription} have been tested with a test driven developlment approach, except for the small classes such as Experiment, Annotation, GeneFile and GenomizerHttpPackage as they are very straight forward (And they are indirectly shown as working through the tests of the other classes).

Most tests are run against the mockup server. Albeit some are run against the real server as the authenticity of both the methods ability to handle data as well as their ability to get a response from the real server is relevant.
