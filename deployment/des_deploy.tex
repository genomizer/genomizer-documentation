{\huge Deploy}
\subsubsection{Swing application releases}
When the \appName\ system is starting to come together like the big system it is, it is important to keep track of the different releases that are made to the customer. Since the project is already hosted on GitHub, the Desktop group decided to use GitHubs Release Tag to have it all collected, see \refer{fig:des_gith-rel}
\begin{figure}[htb]
	\addImage{github_release.png}
	\caption{View of the GitHub site and the release tag.}
	\label{fig:des_gith-rel}
\end{figure}
\cite{des_rel-github}.

The versioning system that was chosen is based on both recommendations from the course supers, and GitHub themselves. The versioning is called Semantic Versioning\cite{des_semver} and has three levels of updates described below;

\begin{quotation}
	Given a version number MAJOR.MINOR.PATCH, increment the:
	\begin{enumerate}
		\item MAJOR version when incompatible API changes has been made,
    	\item MINOR version when you add functionality in a backwards-compatible manner, and
    	\item PATCH version when you make backwards-compatible bug fixes.
	\end{enumerate}
	Additional labels for pre-release and build metadata are available as extensions to the MAJOR.MINOR.PATCH format.
\end{quotation}
Releases are independent from the Sprint deliveries. Patches for fixing minor bugs and deployment of new functionality is quickly distributed to the customer that way, and feedback can come right away.

When a new release is planned, the production code is merged into master(this is done every day as well) and then branched into a separate Release branch. Code that is not production ready is then deleted or commented away. An executable .jar file is then created and tested on the three platforms, based on the test cases written for the implemented functionality. All functionality and notes about the release is documented in a changelog that is then attached to the release. The release is lastly notified by the PO-team to the customer.
