To start the server, java needs to be installed on the computer and a runnable JAR file needs to be created.
There are many ways to create such a file, for example, the terminal or
an IDE like eclipse could be used. \\The server also needs a database to work properly. This guide assumes that a database is present, otherwise a new database needs to be created and currently hardcoded into the server.\\
\\
The creation of the runnable JAR file in the IDE eclipse will be explained in \ref{sec:com_UsingEclipse} below.
\subsection{Using eclipse to create a runnable JAR file}
\label{sec:com_UsingEclipse}
This guide was written 2014-05-09 which means that the process of creating the runnable JAR file with eclipse might have changed slightly, but the main idea should still be valid.\\
\\
To create the runnable JAR file with eclipse, follow these steps:
\begin{enumerate}
\item Open eclipse and import all the code into a project.
\item Rightclick on the project and choose export.
\item Expand the folder "java" and then choose "runnable JAR file".
\item Make or choose an already existing launch configuration where ServeMain is the class containing the main-method.
\item Choose an export location for the runnable JAR file.
\end{enumerate}

\subsection{Starting the server}
Here the actual startup of the server will be explained in a step by step manner.
In order for this to work, the runnable JAR file must have been created.
\begin{enumerate}
\item Choose a computer that should host the server.
\item Make a runnable JAR file of all the code and place it inside a folder on the computer.
\item Start the terminal and navigate to the folder containing the runnable JAR file.
\item In the terminal, type: "java -jar "jarfilename".jar "dbsetting" "portnumber"" (exclude all "). All arguments are explained in more detail in \ref{sec:com_ArgExpl}.\\ 
The image \refer{fig:com_runserverterminal} below is an example of what it could look like when running the command. 
\end{enumerate}
\begin{figure}[h]
\addImage{com_RunServer.png}
\caption{Example of execution of the server}
\label{fig:com_runserverterminal}
\end{figure}

\subsubsection{Argument explenation}
\label{sec:com_ArgExpl}
To start the server, the system administrator needs to take some arguments into consideration before the starting command in the terminal is executed.\\
There are two different arguments that are passed to the server when it is started.
The server uses default settigs if less then two arguments are passed, which currently is the database give from support and port 7001.

\paragraph{dbsetting}
This is the argument that tells the server what kind of database to use.
The argument currenlty has tre choices: global, test and nothing. 

\begin{itemize}
\item If "global" is selected, the server currently uses a database that is located at Umeå universitet in lecture hall MC333.
\item If "test" is the choice, the server used a database that has been given by support at Umeå universitet. 
\end{itemize}

\paragraph{port}
This is the server portnumber: \serverPort\ .
