To start the server, java needs to be installed on the computer and a runnable JAR file needs to be created.
This requires the following things to be installed on the computer: Git, Ant and Java JDK. 
Refer to Appendix \ref{chap:exp_app_debian} on how to install these.
If these are already installed refer to \ref{sec:com_download} on how to download the source files.
\subsection{Downloading the source code}
\label{sec:com_download}
The source code for the Genomizer server is hosted at Github and is completely open source. It can be downloaded in
two ways. Either manually from http://www.github.com/genomizer/genomizer-server where there is a button to
download the entire project as a zip file or using Git from command line in the following way:
\begin{verbatim}
git clone https://github.com/genomizer/genomizer-server.git
\end{verbatim}
This will create a directory named genomizer-server in the current directory.
\subsection{Creating a runnable JAR file}
\label{sec:com_makejar}
\subsubsection{Command line}
When the source code is downloaded (and unzipped if downloaded manually), use the terminal to navigate into
the genomizer-server directory.
\begin{verbatim}
ant jar
\end{verbatim}
A file called server.jar should be created in the same directory.
\subsubsection{Eclipse}
\label{sec:com_UsingEclipse}
To create the runnable JAR file with Eclipse, follow these steps:
\begin{enumerate}
\item Open eclipse and import all the code into a project.
\item Right-click on the project and choose export.
\item Expand the folder "java" and then choose "runnable JAR file".
\item Make or choose an already existing launch configuration where ServerMain is the class containing the main-method.
\item Choose an export location for the runnable JAR file.
\end{enumerate}

\subsection{Starting the server}
Here the actual startup of the server will be explained in a step by step manner.
In order for this to work, the runnable JAR file must have been created.
\begin{enumerate}
\item Choose a computer that should host the server.
\item Make a runnable JAR file of all the code and place it inside a folder on the computer.
\item Start the terminal and navigate to the folder containing the runnable JAR file.
\item In the terminal, type: \emph{java -jar filename.jar}. Server configuration is explained in more detail in Appendix \ref{sec:com_ArgExpl}.\\ 
\end{enumerate}
