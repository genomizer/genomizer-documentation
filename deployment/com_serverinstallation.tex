To start the server, java needs to be installed on the computer and a runnable JAR file needs to be created.
This requires the following things to be installed on the computer: Git, Ant and Java JDK. 
Refer to Appendix \ref{chap:exp_app_ubuntu} and \ref{chap:exp_app_debian} on how to install these.
If these are already installed refer to \ref{sec:com_download} on how to download the source files.
\subsection{Downloading the source code}
\label{sec:com_download}
The source code for the Genomizer server is hosted at Github and is completely open source. It can be downloaded in
two ways. Either manually from http://www.github.com/genomizer/genomizer-server where there is a button to
download the entire project as a zip file or using Git from command line in the following way:
\begin{verbatim}
git clone https://github.com/genomizer/genomizer-server.git
\end{verbatim}
This will create a directory named genomizer-server in the current directory.
\subsection{Creating a runnable JAR file}
\label{sec:com_makejar}
\subsubsection{Command line}
When the source code is downloaded (and unzipped if downloaded manually), use the terminal to navigate into
the genomizer-server directory.
\begin{verbatim}
ant jar
\end{verbatim}
A file called server.jar should be created in the same directory.
\subsubsection{Eclipse}
\label{sec:com_UsingEclipse}
To create the runnable JAR file with Eclipse, follow these steps:
\begin{enumerate}
\item Open eclipse and import all the code into a project.
\item Right-click on the project and choose export.
\item Expand the folder "java" and then choose "runnable JAR file".
\item Make or choose an already existing launch configuration where ServerMain is the class containing the main-method.
\item Choose an export location for the runnable JAR file.
\end{enumerate}

\subsection{Starting the server}
Here the actual startup of the server will be explained in a step by step manner.
In order for this to work, the runnable JAR file must have been created.
\begin{enumerate}
\item Choose a computer that should host the server.
\item Make a runnable JAR file of all the code and place it inside a folder on the computer.
\item Start the terminal and navigate to the folder containing the runnable JAR file.
\item In the terminal, type: \emph{java -jar filename.jar}. Server configuration is explained in more detail in \ref{sec:com_ArgExpl}.\\ 
\end{enumerate}

\subsection{Configuration}
\label{sec:com_ArgExpl}
The server requires some parameters to be set before it can be used. They should be stored in a file called
``settings.cfg'' in the same directory as the server JAR. It can look like this:
\begin{verbatim}
databaseuser     = admin
databasepassword = password
databasehost     = localhost:6000
databasename     = genomizer
publicaddress    = http://www.genomizer.se
apacheport       = 8000
downloadurl      = /download.php?path=
uploadurl        = /upload.php?path=
genomizerport    = 7000
passwordsalt     = genomizer
passwordhash     = 2fd26e9aea528153a865257a723f6d4859e9f6c4a6775c003ae91297f619c6e8
\end{verbatim}
Each setting should be on a separate line, and be separated by a '=' sign. The number of spaces does not matter,
neither does case on the setting names. Case does matter on the setting values however, password and PassWord are
different things.

\subsubsection{Database settings}
The settings databaseuser, databasepassword, databasehost and databasename are all connected to the SQL database,
and should be the same as the ones used when setting up the database. 
\subsubsection{Addresses and ports}
The settings publicaddress, apacheport, downloadurl, uploadurl and genomizerport decide how the clients
connect to the genomizer system. Downloadurl, uploadurl and genomizerport should generally stay the same as the example file,
but publicaddress and apacheport depends on how the server is set up.
\subsubsection{Password handling}
Passwordsalt is used to increase the security of passwords. It is combined with the password and hashed so that the password does not need to be
saved in plaintext. Passwordhash is the result of this operation, and this parameter should not be set manually. There is a script called
changepwd.sh, which when called with the password as parameter will update settings.cfg with the result.
\subsubsection{Flags}
In addition to the settings file, a set of flags may be set when starting the server from terminal. These are:
\begin{itemize}
\item -p [port] \\This flag sets the listening port. 
\item -debug \\If this flag is set the server will print output to terminal for every request and response. Also other outputs are written aswell.
\item -f [file] \\Uses file to read settings instead of the default file settings.cfg.
\item -nri \\If this flag is set the server will not remove inactive users which are logged in on the server.
\end{itemize}