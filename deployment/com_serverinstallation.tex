To start the server, java needs to be installed on the computer and a runnable JAR file needs to be created.
This requires the following things to be installed on the computer: Git, Ant and Java JDK. 
If these are already installed refer to \ref{sec:com_download} on how to download the source files.

\subsection{Downloading the source code}\label{sec:com_download}

The source code for the Genomizer server is hosted at Github and is completely open source. It can be downloaded in
two ways. Either manually from http://www.github.com/genomizer/genomizer-server where there is a button to
download the entire project as a zip file or using Git from command line in the following way:
\begin{verbatim}
git clone https://github.com/genomizer/genomizer-server.git
\end{verbatim}
This will create a directory named genomizer-server in the current directory.

\subsection{Creating a runnable JAR file}\label{sec:com_makejar}

When the source code is downloaded (and unzipped if downloaded manually), use the terminal to navigate into
the genomizer-server directory.
\begin{verbatim}
ant jar
\end{verbatim}
A file called server.jar should be created in the same directory.


\subsection{Starting the server}
Here the actual startup of the server will be explained in a step by step manner.
In order for this to work, the runnable JAR file must have been created.
\begin{enumerate}
\item Choose a computer that should host the server.
\item Make a runnable JAR file of all the code and place it inside a folder on the computer.
\item Start the terminal and navigate to the folder containing the runnable JAR file.
\item In the terminal, type: \emph{java -jar filename.jar}. Server configuration is explained in more detail in Appendix \ref{sec:com_ArgExpl}.\\ 
\end{enumerate}
