Installing the server requires these three programs to exist on the server machine.
\begin{itemize}
	\item \textit{Java JDK}: To run the server.
	\item \textit{Git}: To download the server source code.
	\item \textit{Ant}: To build the server.
\end{itemize}

The source code for the server is hosted by \textit{Github} and can be downloaded using \textit{Git} or using the home page and get it as a \textit{.zip}. 

Downloads using \textit{Git} is done with the help of \texttt{clone}.
\begin{verbatim}
git clone https://github.com/genomizer/genomizer-server.git
\end{verbatim}

After the source code has been downloaded the server needs to be compiled and built into a runnable \texttt{jar}. This is done with the help of \textit{Ant}. The repository has a build script in the directory, and has a command to build a jar.
\begin{verbatim}
ant jar
\end{verbatim}

All that is left before starting the server is editing the settings file in the main directory for the server(same place as the \texttt{jar}). The settings file is named \texttt{settings.cfg}, it is designed as followed.

\begin{verbatim}
# Database settings.

databaseUser        = genomizer
databasePassword    = password
databaseHost        = localhost:5432
databaseName        = genomizer

# Port number on which the server listens for HTTP connections.
genomizerPort       = 7000

# Directory on the server where uploaded files are stored.

fileLocation        = /data/

# Max. number of threads in the processing work pool.

nrOfProcessThreads  = 5

# Paths to various programs that we use.

bowtieLocation      = resources/bowtie/bowtie
picardLocation      = resources/picard-tools/picard.jar
bowtie2Location     = resources/bowtie2/bowtie2
pyicosLocation      = resources/pyicoteo/pyicos

# Address and port of the publicly visible WWW server (if any)
# that is tunnelling requests to the genomizer server.
# 'wwwTunnelPath' is the path on the WWW server that
# requests to genomizer server are mapped to (usually '/api').

wwwTunnelHost       = https://130.239.192.110
wwwTunnelPort       = 4433
wwwTunnelPath       = /api

# Directory used for storing temporary files during uploading.
# Should be on the same partition as 'fileLocation'.
uploadTempDir       = /data/tmp/
\end{verbatim}

All that is left after the settings is correct is to start the server.

\begin{verbatim}
java -jar server.jar [-debug|-p <port> | -f <alternativSettings.cfg> | -nri ]
\end{verbatim}
As seen the server may be started with optional arguments.
\begin{itemize}
	\item \texttt{debug}: Starts the server with a active debug logger. Printing everything to \texttt{stdout}.
	\item \texttt{p}: Alternativ port.
	\item \texttt{f}: Alternativ settings file other then \texttt{settings.cfg}
	\item \texttt{nri}: No remove inactive users. Halts the thread that removes inactive user tokens.
\end{itemize}
