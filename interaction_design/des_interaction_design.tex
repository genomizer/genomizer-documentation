The desktop application is constructed in a topdown approach that separates all the different functionalities into groups. Similar functions will be grouped together to utilize space. The simplicity of the interaction was not fully prioritized since the customers wanted more focus on a large number of functions. There are however a couple of design choices which have been made to simplify the interaction.

The application is built with tabs that simplifies work by letting the user easily switch between different views. Each tab is described by appropriate name and contains related functionality.

The workspace tab is used as a center for easily transferring data between the different tabs. It has easy access to the download, upload and process functions.

The administration tab design is centered around having different views that can be reached from the buttons on the left side of the screen. The Annotations view is where you can add new annotations to the database. This view has a table of all current annotations in the middle of the screen and a toolbar on the right side. Additional functions can be reached from pop up windows when a user clicks on the buttons in the tool bar.
A principle in the design is when the user types in something wrong, an alert (popup) will be shown telling what went wrong and why, for example if the user did not type in a name of the annotation, a popup telling that a annotation needs a name will be shown.  
\FloatBarrier

