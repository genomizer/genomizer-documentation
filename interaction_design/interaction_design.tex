%The interaction design of the clients, what principles has been used. Directed towards developers. Start with subsection here. Android and iOS need to start with subsubsection.

This chapter goes into detail on how the graphical and interactive parts of the clients are designed. It starts with a general view of the interaction design and then divideds into chapters based on the different clients.

\section{General view}
An important aspect of the user interface is the possibility to choose 
Since the Genomizer application is first and foremost about handling important 
data, it is important to allow the user to be in control but still to protect 
and preserve the data that is already in the system.

The workflow of Genomizer is intended to be as natural as possible for 
the users and to be easily integrated into their daily work routine.

An important aspect of the user interface is the possibility to choose

%\footnote{\url{http://en.wikibooks.org/wiki/GUI_Design_Principles}}

\section{Desktop client}
Screen clients use a tab based navigation between views, these tabs are shown at the top of the user interface. The common views in the current system are search, upload and process.

Search results are displayed in a table, experiments can be expanded to reveal the files contained in the experiment. The files in an experiment are grouped by types where each type consists of a row in the table that may be expanded to reveal the files of that type.

The upload view consists of experiment groups. Each experiment group contains a set of input fields for annotation and a list of files added to this experiment. The user may create new experiments in this view or add files to an existing experiment, multiple files may be added to multiple experiments simultaneously.

The base for the process view contains a set of input fields for the parameters that are to be used when processing a file.

\FloatBarrier
\subsection{\term{Windows}/\term{OS X}/\term{Linux} application}
The desktop application is constructed in a topdown approach that separates all the different functionalities into groups. Similar functions will be grouped together to utilize space.
The application is built with tabs that simplifies work by letting the user easily switch between different views. Each tab is described by appropriate name and contains related functionality.

The workspace tab lets the user easily manage files and experiments. It has easy access to the download and process functions.

The administration tab design is centered around to have different views that can be reached from the buttons on the left side of the screen. The Annotations view is where you can add new annotations to the database. This view has a table of all current annotations in the middle of the screen and a toolbar on the right side. Additional functions can be reached from pop up windows when a user clicks on the buttons in the tool bar.
A principle in the design is when the user types in something wrong, an alert (popup) will be shown telling what went wrong and why, for example if the user did not type in a name of the annotation a popup telling that a annotation needs a name will be shown.  
\FloatBarrier



\FloatBarrier
\section{Web application}
Generally the design of the user interface for the web application is an integration of the principles previously described with core design elements of web and the twitter bootstrap element library.

\subsubsection{Layout and Structure}
The structure of the application is in most cases shallow, the navigational depth is usually two steps but sub views with modal views may result in a depth of 3. There are three types of views which are hierarchical in some way, main views contain sub views and modal views, sub views may contain modal views.
\begin{itemize}
	\item \textbf{Main views:}
A main view covers the entire page. The structure among main views is shallow and the user may freely navigate between all main views using the navigation bar. Typically a main view contains a toolbar and a set of panels.
	\item \textbf{Sub views:}
A sub view is a part of a main view. In this case the main view has a vertical navigation bar on the left side used to navigate between sub views, sub views may not be directly navigated outside of its main view. The user may navigate to other main views from a sub view. Except for the sub navigation bar the sub view covers the entire main view, replacing its content.
	\item \textbf{Modal views:}
Modal views “rolls over” the current main view and are used for specialized operations. Modal views can be navigated to using buttons inside main views and sub views. Usually the user will be taken back to the previous view when the modal is closed but navigation in a sequence of modal views could be implemented in the future.
    \item \textbf{Panels:}
Content that belong together is grouped using so bootstrap panels. Main views and sub views should contain one or more panels.
    
    \item \textbf{Toolbars:}
In main views and sub views we use a toolbar at the top of the view where operation controls available to the user are presented.
    
    \item \textbf{Popovers:}
For elements that belong to a view but have no need to be visible at all times are shown in bootstrap popovers. Popovers that do not belong to a specific view may be placed in the nav bar.
    
\end{itemize}

\subsubsection{Colors}
Grayscale colors are mostly used, black or dark gray is used for text, icons and borders while white or light gray is used for backgrounds. Colors of different hues are used to distinguishing elements from each other and to highlight important elements. Colors with high saturation are reserved for smaller elements while colors with lower saturation can be used regardless of elements size. Light gray of varying brightness may also be used to highlight or distinguish elements.

\subsubsection{Icons}
Buttons that perform actions should always contain an icon as well as text so that the experienced user may more quickly desired actions by identifying buttons at a glance instead of having to read the button text. 

\subsubsection{Batching}
For operations performed on objects that there are multiples of e.g. experiments or files, let the user perform these operations on multiple objects at the same time in cases where it makes sense.

\subsubsection{System administration}
The admin page is built up by a number of components: the main view, the side bar, the create annotation view, the edit annotation view and the genome-release view. The first one is the main view which consists of a sidebar and an empty div-tag. The empty div-tag is then replaced with the annotation list view which has a Create new annotation button and a list of the available annotations on the database with an option to edit. 

When the user clicks on for example Create New Annotation, the div tag in the main view is replaced with the create annotation view. The same goes for the Edit buttons on each annotation. This way we only have to render that specific div-tags current information and the sidebar is unaffected. 

The design is made so that the user should be able to avoid mistakes. For example in the create annotation page the user is not able to create an annotation without filling in all the fields. Futher more the field for Items in drop-down list is disabled if the user don't choose Drop-down list as the annotation type. 

In the Edit annotation view the same principles apply, but also there is a Delete Annotation button on this page which will delete the entire annotation from the database.For that reason we decided to ask if the user is sure of this action and of course made the button red.

The back buttons on the different views work as one would expect and the sidebar option Annotations takes the user back to the main adminview.

The sidebar item ''Genome-releases'' takes the administrator to the page for adding and editing genome-releases. This page have the same look and feel as the previous. The delete buttons are red and will prompt a confirmation-popup. 

The ''Select files to upload'' will as expected open the file explorer and the user chooses files according to normal operativesystem standards, then the ''Upload'' button will prompt the user for information about the files such as species and genomeversion before uploading. 


\FloatBarrier
\section{Android}
%\subsection{Interaction Design}

%The design of the Android application is based on the design proposal suggested by the design team and our aim has been to recreate that look and feel. We did, however, find it necessary to take into consideration some of the Android specific design paradigms which distinguish Android applications from other smart phone platforms. For instance, the design put forth by the design group did not include a so called action bar   to the upper part of the user interface which are used for navigation. However, since these are fundamental to the structure of any Android application, we were inclined to include this feature as a substitute for the slide-in menu described in the original design.

%In the following sub-sections, we will attempt to explain our design desisions.

The \appName\ Android application was designed to allow for a quick search of the database while on the move. It also makes it possible to start file conversions in advance so that the data is ready when further work and analysis is to be done. The app will also provide a way to continuously view the status of the users file conversions. 

The application was designed in close collaboration with the \term{iOS} application in order to provide a consistent experience on both plattforms.  We did, however, find it necessary to take into consideration some of the Android specific design paradigms which distinguish Android applications from other smart phone platforms. One of theses paradigms is the actionbar at the top of the screen that provides navigational functions.In this section the layout and design decisions will be described.


\subsection{Login View}
There are two textfields available for the user to type user name and password and a button to click when user is ready to log in. This is a popular layout for many login screens and thus a design many users are familiar with.


\begin{figure}[ht]
\addScaledImage{0.1}{andLogin.png}
\caption{Login View}
\label{fig:and_login}
\end{figure}
\FloatBarrier

\subsection{Search View}
The design illustrated in \refer{fig:and_search} show the search view. The search annotations are displayed in a list and it is easy to learn how to search. Scroll bars are used for multiple options and textfields are used for free text. At the bottom of the view there is a button (not visible on the picture) to press to start the search.

\begin{figure}[ht]
\addScaledImage{0.1}{and_search.png} 
\caption{Search View}
\label{fig:and_search}
\end{figure}
\FloatBarrier

\subsection{Search Results View}
The design illustrated in \refer{fig:and_result} below show the search result view. The result is shown in a list, sorted by experiments. The list displaying search results is large to facilitate usage for user and to take advantage of the screen space. It's easy to learn how to navigate the list. Scrolling is available if the list is long and if the user clicks on an experiment they are redirected to the experiment view displaying more information about that experiment.

\begin{figure}[ht]
\addScaledImage{0.1}{andResult.png} 
\caption{Search Result View}
\label{fig:and_result}
\end{figure}
\FloatBarrier

\subsection{Experiment View}
The design illustrated in \refer{fig:and_experiment} shows more information about a specific experiment. All files for the experiment selected in the search result view is displayed here organised by data type. Checkboxes are commonly used and most users are familiar with how to handle them when making choices and selecting items. The button \click{Send to conversion} will be used to send selected files to the conversion view.

\begin{figure}[ht]
\addScaledImage{0.1}{andExperiment.png} 
\caption{Experiment View}
\label{fig:and_experiment}
\end{figure}
\FloatBarrier

\subsection{Search Settings View}
The design illustrated in \refer{fig:and_search_settings} is showing the view for search settings. This is a way for the user to select annotations to be displayed in the search result view. The user can select annotations by checking the checkbox next to the annotation name and then click the button to save changes. The changes are stored on internal storage and saved between runs of the application. If the user has no special requests it is also possible to use default settings. This functionality gives the users the possibility to design the search result view the way they want to have it which often is appreciated. 

\begin{figure}[ht]
\addScaledImage{0.4}{SearchSettings.png} 
\caption{Search Settings View}
\label{fig:and_search_settings}
\end{figure}
\FloatBarrier

\subsection{Selecting Files View}
The design illustrated in \refer{fig:and_selected} shows the view for selecting files. The view has four tabs, one for each data type and one for results. To make it easy to navigate the user can switch tab by sliding your finger horizontally. There is also the option of clicking the tabs. 

The files are displayed in a list which gives a clear view to the user. At the bottom of the view there is a button with option to what to do with the data files stored. For example in the view for raw data there is a button to click to convert the files into profile data.

\begin{figure}[h]
\addScaledImage{0.4}{and_selectedfiles.JPG} 
\caption{Selecting Files View}
\label{fig:and_selected}
\end{figure}
\FloatBarrier

\subsection{Convert View}
The design illustrated in \refer{fig:and_convert_man} shows the conversion view. This view displays different parameters that needs to be set before starting to convert data files. The view is clear with headlines and hints guiding the user to what parameters that needs to be set that facilitates the work for the user and also prevents errors from occuring. At the bottom of the view there is a button to press to start the conversion when all the parameters are set. 

\begin{figure}[h]
\addScaledImage{0.1}{and_convert.png}
\caption{The Convert View}
\label{fig:and_convert_man} 
\end{figure}
\FloatBarrier


\FloatBarrier
\section{iOS}
Focus has been on making a nice looking application with an intuitive workflow and to follow the iOS design principles. Some of the design decisions are motivated in the text below.

\subsection{Navigation bar}
A navigation bar is used to make access to different main functionalities available at all times. Big and clear icons are used to show the user which view they all represent. It is also possible to simply swipe between the different views to increase the speed of which the advanced user can use the system.

\subsection{Login Screen}
The login screen has two responsibilities; to make a nice first impression and to make it easy for the user to login. The design is kept simple and clean to avoid distractions.

\subsection{Search View}
The search view is designed to be usable for both advanced and new users. A list with available annotations is displayed to make it easy to do basic searches fast. Some annotations can only be selected with a picker view, while others are edited by typing free text. The reason for the occurance of the picker views is to simplify searches and help the user to make correct search requests. For example, the sex of an individual can only be male, female or unknown. Other values for the sex annotation would be nonsence! The search button disappears when no annotation is selected to decrease the chance of user sending empty searches and to increase the understanding of the switches. 

\begin{figure}[ht]
\addTwoImages	{ios_search1.PNG}{a}
		{ios_search2.PNG}{b}
\caption{The search screen.}
\label{fig:ios_search2}
\end{figure}
\FloatBarrier

Each annotation has a corresponding switch button as seen in \refer{fig:ios_search2}a-b. The button determines if the annotation should be included in the search request. This make it easy to make small changes to the search, while not clearing the annotation values.

The advanced user can customize the search query sent to the server. This gives the user the possibility make more complex search queries and possibly make use of already accuried PubMed-search skills.

\subsection{Search Result View}
The main purpose of the search result view is to give an overview of the search results. The challenge with this view was to summarize large amount of information in a small area. The small screen of the iPhone made it impossible to have columns for each annotation. Instead a decision was made to group the files by experiment as seen in \refer{fig:ios_searchResult2}. The table with the experiments will only expand vertically, both when the number of shown annotations and the number of experiments grows. Thus, the user never has to scroll sideways which would be awkward.

\begin{figure}[ht]
\addScaledImage{0.30}{ios_result3.PNG}
\caption{The search result view.}
\label{fig:ios_searchResult2}
\end{figure}
\FloatBarrier

The user can choose which annotations to display in the result view. This gives the user the possibility to only show the annotations which are interesting at the moment. 

The files view (see \refer{fig:ios_files2}a), which is shown when the user selects an experiment, only contains the filename of the files in the specific experiment. The annotations is not shown in this view to avoid information overload and to give the user a good overview of the files. The purpose of the plus-symbol next to each file is to add as many files as the user wish to use when selecting processes. More information can be seen when the circled 'i' to the left of the filename is tapped, an example of this can be seen in \refer{fig:ios_files2}b. 

\begin{figure}[ht]
\addTwoImages	{ios_process_files.jpg}{a}
		{ios_files2.PNG}{b}
\caption{The file view.}
\label{fig:ios_files2}
\end{figure}
\FloatBarrier


% The functionality of the Convert files button can be reached from other views, but was added to this view as well to improve the workflow. Instead of first selecting the files, then going to the Selected files view and initiate the convertion from there, the user can quickly convert files directly from the search results.

%\subsubsection{Selected Files}
%The selected files view can be seen in \refer{fig:ios_selectedFiles2}a. The files are grouped into four categories: raw, profile, region and other. This is done by showing each type of files in its own tab located at the top of the screen. The reason for this is to avoid the possibility to select files of different types since the tasks to perform are file type specific. It also gives a better overview of the files when only one type is shown instead of showing all files at the same time. The files are also grouped into respective experiment they belong to. This is done to avoid confusion of what file belong to which experiment with similar filenames etc. Additionally, the top navigation bar menu is following the iOS design guidelines. 
%
%When at least one file is selected with the switch next to it (see \refer{fig:ios_selectedFiles2}b) a button will appear at the bottom of the screen letting the user choose a task to perform with the file by leading to the select task menu, as seen in \refer{fig:ios_convertParameters}a.
%
%\begin{figure}[ht]
%\addTwoImages	{ios_selected1.PNG}{a}
%		{ios_selected2.PNG}{b}
%\caption{The selected files view.}
%\label{fig:ios_selectedFiles2}
%\end{figure}
%\FloatBarrier


\subsubsection{Create processes}
The view to create a process is focused on the user's current workflow. The user wants to use the selected files from the files view as input and then select a specific process on those files, which creates output-files to be used as input-files in another process. This creates a sequence of processes which the server will execute one process at a time. Moreover each input-file for a process will be executed parallel with eachother. The input-files and the output-files are separated by the process which will be executed on the input-files. To make it easy to understand what will be executed on each step of the sequence the separator between input-files and output-files is the name of the process and an arrow from the input-files to output-files. The color of a output-file will have the same color as its corresponding input-file to track a file's conversion-process, from start to finish, see \refer{fig:ios_creating_process}a-c. 



\begin{figure}[ht]
\addThreeImages {ios_empty_make_process.jpg}{a}
	{ios_one_process.jpg}{b}
	{ios_many_process.jpg}{c}
\caption{Creating a process}
\label{fig:ios_creating_process}
\end{figure}
\FloatBarrier

\subsubsection{Processes}

As visible in \refer{fig:ios_processingStatus}, we chose to build the processing status view with a simple tableview, to make it dynamic and easy. We also think this gives the user the best possible overview of current processes. The status is color coded to make it as easy as possible for the user to see which processes are in which state.
\begin{figure}[h]
\addScaledImage{0.30}{ios_processes1.PNG}
\caption{The process status view.}
\label{fig:ios_processingStatus}
\end{figure}
\FloatBarrier

\subsubsection{Alerts}
Alerts are simple banners animating down from the top to give the user a headsup of what is going wrong and what is going the way it is expected. The user can tap the banner to dismiss it or if nothing is done it will animate away in 2 seconds. The banners were introduced to not stop the user with prompts which the user has to react with to be able to further use the system. A red banner, as seen in \refer{fig:ios_alerts}, indicates an error and a white with a green icon, indicates something went as expected. The colors are used to allow the user to just notice which color pops down and give them somewhat of an understanding of what is going on. without having to read the text every time.
\begin{figure}[ht]
\addScaledImage{0.30}{ios_alert1.PNG}
\caption{Examples of alert}
\label{fig:ios_alerts}
\end{figure}
\FloatBarrier



