\subsection{Process limitations}
\begin{itemize} 
\item Ratio calculation has a limitation that it requires processing to be run on two files and that one of these files needs to be explicitly named in the input.

\item One known problem with the smoothing subprogram, is that if a chromosome is smaller 
than the windowsize, the program will then smooth that chromosome together with 
the following chromosome. In practice this problem should never occur on a regular
file when doing smoothing once. 

However, if stepping is done on a file with a step size of, for example 10 000, and 
we want to smooth the new file again with a window size of 100, then the shortest
chromosome in the original file must be atleast 1 000 000 rows. From what we have seen
of the melanogaster data the shortest chromosome there is roughly 200 000 rows. 

Therefore a user should be cautious when smoothing the second time 
on file that has been stepped with a large step size.  
\end{itemize}
